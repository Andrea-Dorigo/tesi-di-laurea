\chapter{Cenni teorici}

\section{Etimologia e storia della parola}

La parola stigma deriva dal latino stigma (-ătis) con l’accezione di «marchio, macchia, punto», propriamente «puntura», e dal greco \textgreek{στίγμα -ατος}, derivato di \textgreek{στίζω} «pungere, marcare» \autocite{treccani-stigma}.
Nella Grecia antica si usava per indicare il marchio che si imprimeva a fuoco sul bestiame come segno di proprietà e, successivamente, il marchio a fuoco che si imprimeva sulla fronte per punire i delinquenti e gli schiavi fuggitivi \autocite{garzanti-linguistica-stigma}.
Risulta quindi evidente l’originaria connotazione negativa figurata di «marchio d’infamia», che nel corso del Novecento si è poi evoluta, entrando a far parte del linguaggio comune nel senso di “biasimare energicamente, disapprovare con asprezza” \autocite{crusca-stigmatizzare}.
Comunemente la parola stigma indica quei \textit{“... segni fisici che vengono associati agli aspetti insoliti e criticabili della condizione morale di chi li ha”} (Goffman, 1963) e perciò, l’attribuzione di qualità negative a una persona o a un gruppo di persone, soprattutto rivolta alla loro condizione sociale e reputazine (Treccani).
L’assegnazione di questi aggettivi a determinate categorie di persone è frutto dell’abitudine e di un \textit{modus operandi} stabile che viene tramandato di generazione in generazione.
È quindi la società a stabilire quali attributi siano da considerare ordinari e naturali nel definire le persone, determinando a priori la natura del rapporto con loro, e a definirne l’identità sociale, partendo da supposizioni e spesso senza consapevolezza diretta di questo processo (Goffman, 1963).
Attraverso le relazioni sociali e le interazioni all'interno di strutture prestabilite si apprendono, condividono e co-costruiscono le identità sociali, che diventano anticipazioni sociali, aspettative normative condivise sia dal “normale” che dagli stigmatizzati: secondo il pensiero di Goffman (1963), per studiare lo stigma si dovrebbero primariamente analizzare le relazioni interpersonali. Come hanno brillantemente affermato i sociologi Peter Berger e Thomas Luckmann nel testo La realtà come costruzione sociale, ha origine quindi una realtà condivisa da tutti i membri della società, esperita come oggettivamente fattuale e soggettivamente significativa: una realtà che è contemporaneamente data per scontata a priori come effettivamente vera, e donatrice di senso che guida l’interpretazione e l’approccio che il singolo ha nei confronti della vita quotidiana (Berger \& Luckmann, 1966).
Nel complesso, lo stigma è definito come un insieme di stressors quali etichettamento, stereotipizzazione, isolamento, perdita di status sociale e discriminazioni (Link \& Phelan, 2001), innescati da stereotipi negativi che si sono associati in una particolare società (Ritsher \& Phelan, 2004); atteggiamenti, stereotipi e risposte affettive possono tradursi poi in comportamenti negativi (ad esempio, distanza sociale ed evitamento) nei confronti delle persone stigmatizzate (Link \& Phelan, 2001). Lo stigma quindi si verifica quando un marchio identifica la persona tramite caratteristiche indesiderabili che la screditano agli occhi degli altri (Link et al., 2004).
