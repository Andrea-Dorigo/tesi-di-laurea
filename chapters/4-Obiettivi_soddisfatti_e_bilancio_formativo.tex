% \chapter{Valutazione retrospettiva}
%
% % \section{Obiettivi soddisfatti dallo stage}
% %
% % Valutazione oggettiva riguardo il percorso e i risultati raggiunti da esso.
% %
% % \section{Maturazione professionale acquisita}
% %
% % Descrizione delle conoscenze e abilità professionali acquisite grazie al percorso di \textit{stage}.
% % Valutazione del miglioramento personale portato avanti durante il percorso di stage.
% %
% %
% % \section{Distanza tra le competenze necessarie e quelle acquisite nel corso di studi}
% %
% % Breve valutazione riguardo le difficoltà riscontrate, e considerazioni riguardo le competenze ottenute durante il corso di laurea che più mi hanno aiutato durante il percorso.
%
% \section{Obiettivi dello stage raggiunti}
%
% Gli obiettivi principali dello stage sono stati raggiunti con successo.
%
% I microservizi che compongono il prodotto finale hanno raggiunto efficacemente il risultato preposto, creando il sistema richiesto dalla sperimentazione; i due servizi di test quali \sacr{ws} \textit{Client} e \sacr{ws} \textit{Provider} si scambiano messaggi tramite un \middleware\ basato su Apache Kafka.
%
% La sperimentazione ha testato alcune delle capacità di Apache Kafka con esito positivo, fornendo le basi per ulteriori percorsi di approfondimento che possono portare all'implementazione della piattaforma di \textit{event streaming} all'interno degli attuali sistemi di integrazione con il ruolo di \middleware.
%
% Un possibile percorso potrebbe ad esempio simulare un caso d'uso molto più complesso, con simulazione di un flusso di dati continuo e di grandi dimensioni, con dati provenienti da fonti multiple, un numero maggiore di \textit{producer} e \textit{consumer}, e la sperimentazione di ulteriori funzionalità presenti nei \middleware\ attualmente utilizzati.
%
% \bigskip\noindent
% Di seguito viene ripresa parte della tabella \ref{tab:pianificazione} vista nella sezione \ref{sec:pianificazione}.
%
% \onehalfspacing
% \begin{small}
%   \begin{center}
%     \centering
%     \renewcommand\arraystretch{1.6}
%     \begin{longtable}{| >{\centering\arraybackslash}m{2cm}|m{9.5cm}|>{\centering\arraybackslash}m{2.2cm}|}
%       \hline
%       \textsc{\textbf{Requisito}} & \textsc{\textbf{Task associati}} & \textsc{\textbf{Completato}} \\
%       \hline
%       O-5.1 & Analisi dei casi d'uso reali & \textsc{si} \\
%       \hline
%       O-5.2 & Realizzazione dei componenti per l'esecuzione dei casi di test & \textsc{si}\\
%       \Xhline{2\arrayrulewidth}
%       O-6.1 & Analisi re-ingegnerizzazione e collaudo del flusso di integrazione asincrono & \textsc{si} \\
%       \Xhline{2\arrayrulewidth}
%       O-7.1 & Analisi e re-ingegnerizzazione e collaudo del flusso di integrazione asincrono con callback & \textsc{si}\\
%       \Xhline{2\arrayrulewidth}
%       O-8.1 & Analisi e re-ingegnerizzazione e collaudo del flusso di integrazione sincrono & \textsc{no}\\
%       \hline
%       \textsc{Non previsto} & Sperimentazione di funzioni aggiuntive: protezione di un dato sensibile & \textsc{si}\\
%       \hline
%
%       \caption{Obiettivi dello stage raggiunti}
%     \end{longtable}
%   \end{center}
% \end{small}
% % \doublespacing
%
% Come detto in precedenza, il requisito O-8.1 è stato scartato in favore della sperimentazione di alcune funzioni aggiuntive di Kafka, inizialmente non pianificate.
%
% % \section{Prodotti sviluppati}
% %
% % Il prodotto finale del progetto è composto da i diversi componenti illustrati nella sottosezione \ref{sub:uml_component}, per un totale di dieci servizi ognuno nel proprio
% % \textit{container} Docker (conformi con il \textit{deployment diagram} alla sottosezione \ref{sub:uml_deployment}).
% %
% % Questi microservizi hanno raggiunto efficacemente il risultato preposto, creando il sistema richiesto dalla sperimentazione; i due servizi di test quali \sacr{ws} \textit{Client} e \sacr{ws} \textit{Provider} si scambiano messaggi tramite un \middleware\ basato su Apache Kafka.
%
% \section{Contenuti formativi acquisiti}
% \onehalfspacing
% \begin{small}
%   \begin{center}
%     \centering
%     \renewcommand\arraystretch{1.6}
%     \begin{longtable}{| >{\centering\arraybackslash}m{2cm}|m{9.5cm}|>{\centering\arraybackslash}m{2.2cm}|}
%       \hline
%       \textsc{\textbf{Requisito}} & \textsc{\textbf{Task associati}} & \textsc{\textbf{Completato}} \\
%       \hline
%      %  O-1.1 & Incontro con le persone coinvolte nel progetto per discutere i requisiti e le richieste relative al sistema da sviluppare & \textsc{si} \\
%      %  \hline
%      %  O-1.2 & Verifica credenziali e strumenti di lavoro assegnati & \textsc{si}\\
%      %  \hline
%      %  O-1.3 & Presa visione dell’infrastruttura esistente & \textsc{si}\\
%      %  \hline
%      %  D-1.1 & Ripasso approfondito riguardo i seguenti argomenti:
%      %    \smallskip
%      %    \begin{itemize}
%      %       \item Ingegneria del \software;
%      %       \item Sistemi di versionamento;
%      %       \item Architetture \software;
%      %       \item Cenni di \textit{Networking}.
%      %     \end{itemize} &  \textsc{si}\\
%      % \Xhline{2\arrayrulewidth}
%      O-2.1 & Nozioni fondamentali riguardo \sacr{eai} e \sacr{soa} & \textsc{si}\\
%      \hline
%      O-2.2 & Approfondimenti riguardo le Architetture a Messaggio, in particolare:
%        \begin{itemize}
%           \item \textit{Integration Styles};
%           \item \textit{Channel Patterns};
%           \item \textit{Message Construction Patterns};
%           \item \textit{Routing Patterns};
%           \item \textit{Transformation Patterns};
%           \item \textit{System Management Patterns}.
%         \end{itemize} & \textsc{si}\\
%     \Xhline{2\arrayrulewidth}
%
%     O-3.1 & Apache Kafka:
%       \begin{itemize}
%           \item Introduzione a Kafka;
%           \item Concetti fondamentali di Kafka;
%           \item Avvio e \sacrfoot{cli};
%           \item Programmazione in Kafka con Java.
%         \end{itemize} & \textsc{si}\\
%     \hline
%     D-3.1 & Esempi e applicazioni di Apache Kafka & \textsc{si} \\
%     \Xhline{2\arrayrulewidth}
%
%     O-4.1 & Confluent Platform:
%       \begin{itemize}
%           \item \textit{Service registry};
%           \item \sacr{rest} \textit{proxy};
%           \item kSQL;
%           \item Confluent \textit{connectors};
%           \item \textit{Control center}.
%       \end{itemize} & \textsc{si}\\
%     \Xhline{2\arrayrulewidth}
%
%
%       \caption{Obiettivi dello stage raggiunti}
%     \end{longtable}
%   \end{center}
% \end{small}
% % \doublespacing
%
% \section{Obiettivi personali raggiunti}
% \section{Distanza dai contenuti del corso di studi}
