\chapter{Contesto aziendale}

\section{Dominio applicativo}
% Breve introduzione al settore del \textit{Enterprise Application Integration}, al tipo di clientela (ovvero pubblica e privata di grandi dimensioni, big data), alla tipologia di \textit{software} prodotti dall’azienda per la clientela (\textit{Middleware}), e alla propensione all’innovazione (richieste da parte della clientela
In conclusione al percorso di studi del corso di laurea in Informatica ho effettuato lo \textit{stage} presso \textit{Sync Lab}.
Questa è un'azienda di produzione software e integrazione di sistemi che fornisce principalmente prodotti per clienti di grande dimensione, sia pubblici che privati.

L'azienda è suddivisa in diversi settori con diverse sedi; l'esperienza personale mi ha portato a conoscere il settore dell'\textit{Enterprise Architecture Integration} e del \textit{Tecnical Professional Services Padova} nella sede aziendale di Padova.
Il percorso di \textit{stage} che ho intrapreso è associato al primo di questi, che si occupa principalmente dell'EAI (\textit{Enterprise Application Integration}) ovvero dell'integrazione funzionale di applicazioni aziendali per una clientela di grandi dimensioni, tramite sistemi di integrazione \textit{Middleware}.

I \textit{Middleware} prodotti comprendono l'utilizzo di molteplici linguaggi e tecnologie in continua evoluzione; è un contesto con un'importante propensione all'innovazione, talvolta esplicitamente richiesta dai clienti.

\section{Strumenti organizzativi e processi interni}

Esposizione delle norme organizzative (\textit{online meeting}, \textit{smart working}, presenze in sede), degli strumenti utilizzati nel rapporto con l’azienda (chat, email e \textit{Project Board}), e delle norme di progetto.

\bigskip\noindent
% (DA RIMUOVERE) Breve presentazione delle persone coinvolte nel percorso di stage.
Processi interni in cui sono stato coinvolto: Sviluppo, Collaudo, Verifica, Formazione, Manutenzione/Evoluzione.

\bigskip\noindent
Breve presentazione dei ruoli delle persone coinvolte nel percorso di \textit{stage}.

\section{L’ambiente di lavoro}

Sviluppo indipendente dal sistema operativo, produzione di \textit{software} non strettamente legati ad uno specifico linguaggio, utilizzo di ambienti virtuali quali \textit{Virtual Machine} e \textit{container} per simulare sistemi indipendenti.
