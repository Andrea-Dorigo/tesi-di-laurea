\chapter{Apache Kafka in caso d’uso simulato}
%
% \section{Definizione di un \textit{way of working}}
%
% % Descrizione del Piano di Lavoro presentato, le strategie e il metodo di lavoro stabilito.
% All'inizio del percorso ho delineato un \textit{way of working}, ovvero un metodo di lavoro da mantenere per tutta la durata dello \textit{stage}, insieme al tutor aziendale, il responsabile del settore \sacr{eai} e gli esperti del settore.
%
% Per mantenere un buon livello organizzativo, quantificare l'avanzamento e rendere agevole la verifica e il supporto tecnico l'azienda ha proposto l'utilizzo di una \textit{board} di progetto.
% Tra le tante opzioni disponibili, mi è stata proposta la piattaforma \textit{ClickUp}.
% Rispetto alla concorrenza questa \textit{board} è ricca di funzionalità, pulita nell'esposizione dello stato del progetto, e la maggior parte delle sue funzioni sono gratuite.
%
% All'inizio del percorso il tutor aziendale e il responsabile del \sacr{eai} hanno creato delle \textit{card} contenenti le attività previste per ogni settimana (\textit{task}) al fine di fornire una struttura generale del progetto.
% All'interno di questi \textit{task} vi sono i concetti chiave, attività previste e obiettivi settimanali che lo stagista è tenuto a seguire per garantire l'efficacia del prodotto finale.
% Oltre a questi \textit{task} principali, ho potuto creare di \textit{task} ausiliari e dei \textit{subtask} per descrivere più adeguatamente l'attività in corso.
%
% Ciascun \textit{task} contiene una colonna laterale dove ho mantenuto un \textit{log} di tutto ciò che è stato eseguito relativo al \textit{task} in questione, allo scopo di esplicitarne il progresso e rendere agevole un eventuale supporto dal tutor o l'evoluzione futura.
%
% Ogni settimana è previsto un \textit{online meeting} per la verifica del progresso ove necessario, la risposta ad eventuali questioni sollevate, e spiegazioni riguardo lo sviluppo della settimana successiva.
% Alcune di queste videoconferenze ha visto la partecipazione di altri esperti che mi hanno aiutato a comprendere meglio il caso d'uso da reingegnerizzare, riassumendo lo stato attuale del sistema d'integrazione per uno dei clienti con relativi \textit{file} utilizzati.
%
% Per mantenere alto il livello di organizzazione, efficienza ed efficacia, all'inizio di ogni giornata lavorativa ho creato un breve piano giornaliero con successivo consuntivo a fine giornata.
% Questo ha permesso al tutor di verificare rapidamente il corretto avanzamento del processo in corso e a me di mantenere il \textit{focus} su di esso.
%
% \section{Formazione}
%
% Il processo di Formazione ha avuto un importante ruolo all'interno dello \textit{stage}, con una durata complessiva di circa quattro settimane.
% La causa di questo lungo periodo è data dallo studio di diversi ambiti e concetti, in particolare il settore del \textit{\acrlong{eai}} e la nuova tecnologia di Kafka.
%
% Durante questo processo Sync Lab mi ha fornito del materiale didattico per l'apprendimento, quali diapositive e appunti di origine aziendale e l'accesso a dei corsi riguardo \textit{Software architecture}, \textit{\acrlong{soa}} (tramite i corsi online su \textit{Coursera}) e Apache Kafka (tramite i corsi online su \textit{Udemy}).
%
% Il tutor aziendale e responsabile \sacr{eai} hanno fornito durante i \textit{meeting} settimanali ulteriori chiarimenti e approfondimenti sul come Sync Lab applica questi concetti nello sviluppo di architetture \software.
%
% Per i corsi online a maggiore contenuto nozionistico ho redatto degli appunti riassuntivi, con lo scopo di consolidare l'apprendimento e velocizzare la verifica del tutor aziendale.
%
% \section{Analisi di un caso d'uso reale e sviluppo di un caso d'uso simulato}
%
% Ad alcune videoconferenze ha partecipato anche un altro esperto \textit{senior} aziendale, esterno al progetto di \stage\ in questione, per illustrarmi un caso d'uso in cui l'azienda ha esposto un prototipo di sistema di integrazione utilizzando i concetti di \textit{Web Service}, \sacrfoot{soap} e \textit{request/response}, permettendomi la visualizzazione dei file \sacrfoot{wsdl}, \sacrfoot{xml} e \sacrfoot{xsd} associati.
% Ho pertanto generato un caso d'uso adatto agli scopi dello \stage\ ispirandomi al caso d'uso reale illustrato.
%
% Il caso d'uso modellato tratta una richiesta di credito telefonico da parte di un cliente ad un'azienda di telecomunicazioni tramite \textit{Web Service}.
% La richiesta avviene tramite flusso di un file \sacrfoot{json} che viene trasmesso attraverso i vari servizi che compongono il sistema di integrazione.
% Va precisato che il contenuto di tale \sacr{json} non è strettamente rilevante allo sviluppo e funzionamento del \textit{Middleware}, ma aiuta a stabilire il contesto di utilizzo.
%
% \noindent
% Il caso d'uso è composto dai seguenti passaggi:
% \begin{enumerate}
%   \item il cliente (\textit{\sacrfoot{ws} Client}) effettua una richiesta di credito tramite invio di un file \sacr{json} al successivo Servizio Web in ascolto.
%   \item un servizio composto da \sacrfoot{rest} \sacr{ws} e \textit{Request Producer} riceve il \sacr{json} e lo inserisce in Kafka tramite l'apposito \textit{Kafka Producer}.
% \end{enumerate}
%
% \section{Progettazione architetturale}
%
% \section{\textit{Setup} dell'ambiente di lavoro}
%
%
% \section{Sviluppo}
%
%
% \section{Collaudo}
%
% Confronto con i requisiti posti nel piano di lavoro, valutazione riguardo i risultati raggiunti, quelli non raggiunti e i possibili sviluppi futuri
