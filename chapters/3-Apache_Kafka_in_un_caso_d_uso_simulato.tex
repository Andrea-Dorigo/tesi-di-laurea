\chapter{Apache Kafka in caso d’uso simulato}
%
% \section{Definizione di un \textit{way of working}}
%
% % Descrizione del Piano di Lavoro presentato, le strategie e il metodo di lavoro stabilito.
% All'inizio del percorso ho delineato un \textit{way of working}, ovvero un metodo di lavoro da mantenere per tutta la durata dello \textit{stage}, insieme al tutor aziendale, il responsabile del settore \sacr{eai} e gli esperti del settore.
%
% Per mantenere un buon livello organizzativo, quantificare l'avanzamento e rendere agevole la verifica e il supporto tecnico l'azienda ha proposto l'utilizzo di una \textit{board} di progetto.
% Tra le tante opzioni disponibili, mi è stata proposta la piattaforma \textit{ClickUp}.
% Rispetto alla concorrenza questa \textit{board} è ricca di funzionalità, pulita nell'esposizione dello stato del progetto, e la maggior parte delle sue funzioni sono gratuite.
%
% All'inizio del percorso il tutor aziendale e il responsabile del \sacr{eai} hanno creato delle \textit{card} contenenti le attività previste per ogni settimana (\textit{task}) al fine di fornire una struttura generale del progetto.
% All'interno di questi \textit{task} vi sono i concetti chiave, attività previste e obiettivi settimanali che lo stagista è tenuto a seguire per garantire l'efficacia del prodotto finale.
% Oltre a questi \textit{task} principali, ho potuto creare di \textit{task} ausiliari e dei \textit{subtask} per descrivere più adeguatamente l'attività in corso.
%
% Ciascun \textit{task} contiene una colonna laterale dove ho mantenuto un \textit{log} di tutto ciò che è stato eseguito relativo al \textit{task} in questione, allo scopo di esplicitarne il progresso e rendere agevole un eventuale supporto dal tutor o l'evoluzione futura.
%
% Ogni settimana è previsto un \textit{online meeting} per la verifica del progresso ove necessario, la risposta ad eventuali questioni sollevate, e spiegazioni riguardo lo sviluppo della settimana successiva.
% Alcune di queste videoconferenze ha visto la partecipazione di altri esperti che mi hanno aiutato a comprendere meglio il caso d'uso da reingegnerizzare, riassumendo lo stato attuale del sistema d'integrazione per uno dei clienti con relativi \textit{file} utilizzati.
%
% Per mantenere alto il livello di organizzazione, efficienza ed efficacia, all'inizio di ogni giornata lavorativa ho creato un breve piano giornaliero con successivo consuntivo a fine giornata.
% Questo ha permesso al tutor di verificare rapidamente il corretto avanzamento del processo in corso e a me di mantenere il \textit{focus} su di esso.
%
% \section{Formazione}
%
% Il processo di Formazione ha avuto un importante ruolo all'interno dello \textit{stage}, con una durata complessiva di circa quattro settimane.
% La causa di questo lungo periodo è data dallo studio di diversi ambiti e concetti, in particolare il settore del \textit{\acrlong{eai}} e la nuova tecnologia di Kafka.
%
% Durante questo processo Sync Lab mi ha fornito del materiale didattico per l'apprendimento, quali diapositive e appunti di origine aziendale e l'accesso a dei corsi riguardo \textit{Software architecture}, \textit{\acrlong{soa}} (tramite i corsi online su \textit{Coursera}) e Apache Kafka (tramite i corsi online su \textit{Udemy}).
%
% Il tutor aziendale e responsabile \sacr{eai} hanno fornito durante i \textit{meeting} settimanali ulteriori chiarimenti e approfondimenti sul come Sync Lab applica questi concetti nello sviluppo di architetture \software.
%
% Per i corsi online a maggiore contenuto nozionistico ho redatto degli appunti riassuntivi, con lo scopo di consolidare l'apprendimento e velocizzare la verifica del tutor aziendale.
%
% \section{Progettazione di un caso d'uso}
%
% Ad alcune videoconferenze ha partecipato anche un altro esperto \textit{senior} aziendale, esterno al progetto di \stage\ in questione, per illustrarmi un caso d'uso in cui l'azienda ha esposto un prototipo di sistema di integrazione utilizzando i concetti di \textit{Web Service}, \sacrfoot{soap} e \textit{request/response}, permettendomi la visualizzazione dei file \sacrfoot{wsdl}, \sacrfoot{xml} e \sacrfoot{xsd} associati.
% Ho pertanto generato un caso d'uso adatto agli scopi dello \stage\ ispirandomi al caso d'uso reale illustrato.
%
% Il caso d'uso modellato tratta una richiesta di credito telefonico da parte di un cliente ad un'azienda di telecomunicazioni tramite \textit{Web Service}, per soddisfare il requisito dello sviluppo della reingegnerizzazione del flusso di dati asincrono.
% La \textit{request} avviene tramite flusso di un file \sacrfoot{json} che viene trasmesso attraverso i vari servizi che compongono il sistema di integrazione, basatosul \textit{Design Pattern} di tipo \textit{publish/subscribe}.
%
% Va precisato che il contenuto di tale \sacr{json} non è strettamente rilevante allo sviluppo e funzionamento del \textit{Middleware}, ma aiuta a stabilire il contesto di utilizzo.
%
% \bigskip
% \begin{figure}[h]
%   \begin{center}
%     \includegraphics[width=\textwidth, trim={0.08cm 0 0 0.08cm},clip]{images/uc_sequence.png}
%     \caption{Diagramma di sequenza \sacr{uml} per il prototipo di caso d'uso iniziale}
%     \captionsetup{aboveskip=2pt}
%     \caption*{\begin{footnotesize}\textit{Fonte: elaborazione personale}\end{footnotesize}}
%   \end{center}
% \end{figure}
%
% \noindent
% Il caso d'uso (figura \thefigure) è composto dai seguenti passaggi:
% \begin{enumerate}
%   \item il cliente (\textit{\sacrfoot{ws} Client}) effettua una richiesta di credito tramite invio di un file \sacr{json} al successivo Servizio Web in ascolto.
%   \item il servizio composto da \sacrfoot{rest} \sacr{ws} e \textit{Request Producer} riceve il \sacr{json} e lo inserisce in Kafka tramite l'apposito \textit{Kafka Producer}, assumendo la funzione di \textit{publisher}.
%   \item il servizio di \textit{Request Consumer}, sottoscritto al \textit{topic} in questione riceve il \sacr{json} e lo invia al \sacr{ws} finale tramite una \sacr{rest} request.
%   \item il servizio in coda chiamato \sacr{ws} \textit{Provider} riceve il \sacr{json}; grazie ai dati ricevuti è in grado di fornire il servizio richiesto dal \textit{Client} nello \textit{step} 1.
% \end{enumerate}
%
% La modellazione dell'architetture e struttura del sistema da sviluppare seguirà questo prototipo di \sacrfoot{uc}.
% Il modello associato al caso asincrono con \textit{callback} seguirà la stessa struttura e \textit{step} dello \sacr{uc} illustrato qui sopra, con l'aggiunta speculare del messaggio di ritorno.
%
% \section{Progettazione architetturale}
%
% La progettazione architetturale ha portato alla produzione di diversi diagrammi \sacr{uml} per rappresentare efficacemente l'architettura del prodotto e fornire un modello da seguire durante il processo di sviluppo.
% Il processo ha richiesto frequenti \textit{meeting} e confronti per raggiungere un risultato finale soddisfacente al fine della sperimentazione.
% % I diagrammi di maggiore rilevanza, oltre ad essere esposti all'interno di questa sezione accanto alla spiegazione associata, sono allegati in formato più grande a
% I diagrammi rappresentano i componenti \textit{color coded}, notazione utilizzata per dare continuità e chiarezza attraverso le diverse tipologie di \sacr{uml} \textit{diagrams}.
%
% \subsubsection{\sacr{uml} \textit{sequence diagrams}}
%
%
% A partire dallo \sacr{uc} descritto nela sezione precedente, ho prodotto un \sacr{uml} \textit{sequence diagram} più approfondito per rappresentare il flusso del \sacr{json} tra i vari componenti (figura seguente).
% \bigskip
% \begin{figure}[h]
%   \begin{center}
%     \includegraphics[width=\textwidth, trim={0.2cm 0 0 0.2cm},clip]{images/a_sequence2.png}
%     \caption{Diagramma \sacr{uml} di sequenza per la reingegnerizzazione del flusso asincrono}
%     \captionsetup{aboveskip=2pt}
%     \caption*{\begin{footnotesize}\textit{Fonte: elaborazione personale}\end{footnotesize}}
%   \end{center}
% \end{figure}
%
% La progettazione del sistema associato al caso asincrono con \textit{callback} comprende il flusso descritto qui sopra, con aggiunta del flusso di ritorno descritto nella figura seguente.
% \bigskip
% \begin{figure}[h]
%   \begin{center}
%     \includegraphics[width=\textwidth]{images/ac_sequence.png}
%     \caption{Diagramma di sequenza \sacr{uml} per la reingegnerizzazione del flusso asincrono con \textit{callback}}
%     \captionsetup{aboveskip=2pt}
%     \caption*{\begin{footnotesize}\textit{Fonte: elaborazione personale}\end{footnotesize}}
%   \end{center}
% \end{figure}
%
% La reingegnerizzazione del flusso sincrono è stata scartata in favore dello studio di funzionalità aggiuntive tramite l'utilizzo della piattaforma di \textit{event streaming}.
% La progettazione di un sistema basato su questo flusso era inizialmente prevista (come requisito desiderabile) nel piano di lavoro iniziale poiché associata ad un caso d'uso reale (di cui si è parlato nelle sezioni precedenti), ma infine è stata giudicata poco opportuna e fuori dagli scopi di Apache Kafka, un sistema basato sull'asincronicità
%
% Il tempo associato a tale requisito è stato pertanto riproposto per testare un'altra funzione utile in un \middleware\, quali la trasformazione di alcuni dati presenti nel \sacr{json}.
% Più precisamente, è stato aggiunto un dato sensibile che viene nascosto e sostituito con asterischi "*" dopo la produzione del \textit{topic} in Kafka grazie all'utilizzo di Kafka Streams.
% \bigskip
% \begin{figure}[h]
%   \begin{center}
%     \includegraphics[width=\textwidth]{images/ap_sequence.png}
%     \caption{Diagramma di sequenza \sacr{uml} per la reingegnerizzazione del flusso asincrono con protezione dei dati sensibili.}
%     \captionsetup{aboveskip=2pt}
%     \caption*{\begin{footnotesize}\textit{Fonte: elaborazione personale}\end{footnotesize}}
%   \end{center}
% \end{figure}
%
% In figura \thefigure\ possiamo vedere il nuovo flusso asincrono con protezione (mascheramento) del dato sensibile.
% La versione con \textit{callback} del flusso qui sopra prevede anche il flusso di ritorno (\textit{callback}), non illustrato in figura ma facilmente intuibile grazie al grafico che la precede.
%
% \subsubsection{\sacr{uml} \textit{deployment diagrams}}
%
% A supporto di questi \sacr{uml} \textit{sequence diagram} che rappresentano efficacemente il flusso di dati, punto focale dell'intero sistema di integrazione (asincrono con la protezione del dato sensibile), ho prodotto ulteriori diagrammi, tra i quali il \textit{deployment diagram}.
% Questo diagramma ha lo scopo di rappresentare la configurazione dei processi \textit{run time}, modellando la struttura di base in cui eseguono i diversi servizi;
% il diagramma esprime l'ambiente in cui i vari componenti risiedono e ove essi comunicano tra di loro.
%
% Con l'approvazione degli esperti aziendali, ho deciso di appoggiare il sistema di integrazione sulla piattaforma Docker.
%
% \begin{figure}[h]
%   \begin{center}
%     \includegraphics[width=\textwidth,  trim={0 0.2cm 0.2cm 0},clip]{images/ap_deployment.png}
%     \caption{\sacr{uml} \textit{deployment diagram} per la reingegnerizzazione del flusso asincrono (protetto)}
%     \captionsetup{aboveskip=2pt}
%     \caption*{\begin{footnotesize}\textit{Fonte: elaborazione personale}\end{footnotesize}}
%   \end{center}
% \end{figure}
%
% Il diagramma di \textit{deployment} (figura \thefigure) vede pertanto l'utilizzo di numerosi \textit{container} indipendenti che dialogano attraverso una rete locale all'interno di Docker.
% Questi \textit{container} sono raffigurati dai vari nodi (rappresentati dai cubi in rilievo in figura).
% A questa notazione fa eccezzione il nodo virtuale intitolato "Kafka Cluster", che ha solamente lo scopo di raggruppare i vari nodi legati al \textit{environment} di Kafka con funzione comune, ma che in realtà non compone un container reale a se stante.
% All'interno di questi nodi sono rappresentati gli artefatti che eseguono nel relativo \textit{container}, per esplicitare la presenza dei componenti.
% Si può inoltre notare che l'ambiente di Apache Kafka è composto da un \textit{cluster} composto da due servizi \textit{Broker} e tre servizi \textit{Zookeeper}, allo scopo di simulare un caso d'uso reale in cui i diversi componenti sono distribuiti in sistemi indipendenti e garantiscono l'affidabilità dello \textit{streaming} di eventi.
%
% \subsubsection{\sacr{uml} \textit{component diagram}}
%
% \begin{figure}[h]
%   \begin{center}
%     \includegraphics[width=\textwidth]{images/ap_component.png}
%     \caption{\sacr{uml} \textit{component diagram} per la reingegnerizzazione del flusso asincrono (protetto)}
%     \captionsetup{aboveskip=2pt}
%     \caption*{\begin{footnotesize}\textit{Fonte: elaborazione personale}\end{footnotesize}}
%   \end{center}
% \end{figure}
%
% Allo scopo di riassumere elegantemente i vari componenti del sistema ho elaborato un \sacr{uml} \textit{component diagram}.
%
%
% % , basati su delle immagini Linux leggere e personalizzate.
%
% % Come si può vedere dalla figura \thefigure,
%
% \section{\textit{Setup} dell'ambiente di lavoro}
%
%
% \section{Sviluppo}
%
%
% \section{Collaudo}
%
% Confronto con i requisiti posti nel piano di lavoro, valutazione riguardo i risultati raggiunti, quelli non raggiunti e i possibili sviluppi futuri
