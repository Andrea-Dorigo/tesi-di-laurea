\chapter{Apache Kafka nell’Integrazione Aziendale}

% \section{L'evoluzione delle architetture di integrazione}
\section{Obiettivi aziendali}
\subsection{Migrazione verso un'\textit{Event Driven Architecture}}

Per soddisfare le richieste dei clienti ed essere sempre competitiva e all'avanguardia, una priorità di Sync Lab sono le esplorazioni tecnologiche e di prodotto anche tramite l'utilizzo di percorsi di \stage\ insieme ai laureandi, come quanto accaduto nella mia esperienza.
Questi percorsi consentono all'azienda non solo di testare l'utilizzo di nuovi \software\ ma anche di conoscere e mettere alla prova le capacità del laureando in vista di una potenziale assunzione al termine dello \stage.
% \bigskip
% Introduzione al motivo aziendale per cui è nato questo percorso di stage: la propensione odierna alle architetture a microservizi, la gestione di grandi flussi di dati in modo efficiente ed eff1icace, una richiesta di un sistema innovativo da parte della clientela.3

Nel settore dell'\gls{g_eai}, l'evoluzione tecnologica è diretta verso soluzioni sempre più distribuite e con un flusso di dati in continuo aumento.
Uno degli obiettivi specifici nell'area \sacr{eai} di Sync Lab è pertanto quello di trovare un \software\ o tecnologia in grado di soddisfare i bisogni dei clienti di gestire un flusso di dati di dimensioni molto maggiori a quelle attuali, tramite architetture a messaggio che utilizzano servizi distribuiti.

\begin{figure}[h]
  \begin{center}
    \includegraphics[width=0.8\textwidth]{images/p2p.png}
    \caption{Illustrazione di un sistema basato sul \sacr{p2p}}
    \captionsetup{aboveskip=2pt}
    \caption*{\begin{footnotesize}\textit{Fonte:} \url{https://news.pwc.be/messaging-architecture-with-salesforce/}\end{footnotesize}}
  \end{center}
\end{figure}

Nei \middleware\ per i sistemi di integrazione, l'aumento del flusso di dati e lo spostamento verso strutture distribuite provoca una difficoltà nella trasmissione dei dati tra i diversi servizi.
Nell'ambito dell'integrazione per un cliente di piccole dimensioni, in cui i dati circolano tra un numero di componenti limitato, può essere sufficiente un'architettura di tipo \sacrfoot{p2p}.

Nel caso di un cliente di maggiori dimensioni tuttavia questo approccio rende la manutenzione e gestione del flusso di dati molto difficoltoso e costoso in termini di risorse (figura \thefigure), dato che il grande numero di collegamenti tra i vari punti.

Una delle soluzioni che viene maggiormente implementata per risolvere questo problema è la migrazione verso una \sacr{eda} (\textit{\acrlong{eda}}), un'architettura basata sugli eventi in grado di scambiare dati tra punti multipli.

\begin{figure}[h]
  \begin{center}
    \includegraphics[width=0.8\textwidth]{images/eda.png}
    \caption{Illustrazione di un sistema basato sulla \sacr{eda}}
    \captionsetup{aboveskip=2pt}
    \caption*{\begin{footnotesize}\textit{Fonte:} \url{https://news.pwc.be/messaging-architecture-with-salesforce/}\end{footnotesize}}
  \end{center}
\end{figure}

Questo tipo di architettura è pertanto definita per gestire la produzione, il rilevamento e la reazione agli eventi (figura \thefigure) grazie ad un \textit{Design Pattern} di tipo \textit{Publish/Subscribe}, eliminando i problemi visti precedentemente causati dal sistema \sacr{p2p}.
Questa architettura prevede l'utilizzo di servizi chiamati \textit{Producer}, il cui scopo è fornire dati al \textit{event bus} centrale.
Una volta che i dati sono inseriti all'interno del \textit{bus} centrale, ogni servizio in ascolto (\textit{Subscriber}) li riceverà idealmente in tempo reale.
% \section{L’evoluzione delle architetture di integrazione}
%
% % Introduzione al motivo aziendale per cui è nato questo percorso di stage: la propensione odierna alle architetture a microservizi, la gestione di grandi flussi di dati in modo efficiente ed eff1icace, una richiesta di un sistema innovativo da parte della clientela.3
% L'evoluzione del settore dell'\textit{Enterprise Application Integration} verso soluzioni sempre più distribuite e con un flusso di dati in continuo aumento ha sviluppato nei clienti (e di conseguenza nell'azienda) un interesse verso il prodotto Apache Kafka.
% Il software ha dimostrato negli anni recenti un notevole successo in diversi campi; l'azienda ha interesse nel testare le capacità di Kafka nel soddisfare le esigenze dell'integrazione aziendale.
%
% \bigskip
% \begin{figure}[h]
%   \includegraphics[width=\textwidth]{images/kafka.png}\\
%   \caption{Illustrazione di un sistema a servizi con Apache Kafka}
%   \captionsetup{aboveskip=2pt,font=it}
%   \caption*{Fonte: https://kafka.apache.org/20/documentation.html}
% \end{figure}
%
% Kafka è una piattaforma di \textit{event streaming}, un sistema moderno e distribuito basato sugli eventi anzichè su di una soluzione più classica come può essere quella del \textit{request/response}.
% L'adozione del \software\ nell'ambito dell'EAI è in crescita dato le dimostrate qualità nel gestire grandi moli di dati: la sua performance, sicurezza e scalabilità sono i punti che hanno portato il software al suo attuale successo.
%
% % \bigskip\noindent
% % Esposizioni delle ragioni personali che hanno portato alla scelta di tale percorso.


\subsection{Kafka come Middleware}

Per soddisfare le esigenze di innovazione l'azienda ha avviato un percorso per indagare le capacità del software Apache Kafka nell'ambito dell'integrazione aziendale.

\begin{figure}[h]
  \begin{center}
    \includegraphics[width=0.45\textwidth, trim={0 0 11.5cm 0},clip]{images/kafka2.png}
    \caption{Illustrazione di Apache Kafka in un caso d'uso esemplificativo}
    \captionsetup{aboveskip=2pt}
    \caption*{\begin{footnotesize}\textit{Fonte:} \url{https://iotbyhvm.ooo/apache-kafka-a-distributed-streaming-platform/}\end{footnotesize}}
  \end{center}
\end{figure}

Apache Kafka si integra ottimamente in molti sistemi basati sul \textit{messaging pattern} e una \sacr{eda}, in cui lo scambio affidabile di dati tra numerosi servizi in tempo reale è essenziale (in figura \thefigure\ è illustrato un caso d'uso esemplificativo di un sistema distribuito basato su Kafka).

% \begin{figure}[h]
%   \begin{center}
%     \includegraphics[width=0.35\textwidth, trim={0.07cm 0 0 0},clip]{images/microservices.png}
%     \caption{Illustrazione di un sistema a microservizi con comunicazioni tramite \sacr{api}}
%     \captionsetup{aboveskip=2pt}
%     \caption*{\begin{footnotesize}\textit{Fonte:} \url{https://medium.com/@riyaznet/building-serverless-microservices-on-aws-3959a93c2549}\end{footnotesize}}
%   \end{center}
% \end{figure}

L'interesse di Sync Lab nel \software\ risiede dunque nell'utilizzo di Kafka come un  \textit{Middleware} per soddisfare i problemi di integrazione aziendale e re-ingegnerizzare i flussi di dati preesistenti, ovvero sviluppare un nuovo sistema che consenta la comunicazione tra differenti servizi con un rapido flusso di dati fra di essi.


L'azienda ha avviato un percorso per testare le capacità di Kafka rispetto agli attuali strumenti utilizzati nel settore, per valutare i vantaggi e svantaggi che l'adozione di tale \software\ può fornire al cliente.

\noindent
Sono numerosi i vantaggi che Kafka può portare nel settore, fra cui:
\begin{itemize}
  \item gestione rapida e performante di un enorme flusso di dati;
  \item scalabilità;
  \item sicurezza riguardo la persistenza dei dati;
  \item semplice integrazione e affiancamento a sistemi già esistenti;
  \item l'essere una piattaforma \textit{open source};
  \item processazione dei dati in tempo reale integrata.
\end{itemize}

% \bigskip\noindent
% TODO: Come Kafka possa risolvere i problemi e le necessità viste qui sopra.

\section{Motivazioni e obiettivi personali}

\subsection{Scelta del percorso}

Una delle ragioni che mi ha portato a scegliere questo percorso di \stage\ è l'interesse verso Apache Kafka.
L'utilizzo della piattaforma di \textit{event streaming} è sempre più in crescita, come l'evoluzione verso sistemi sempre più distribuiti e a microservizi.

Un altro fattore fondamentale alla scelta del percorso sono stati la famigliarità con l'azienda, il personale giudizio positivo che ho avuto riguardo il loro metodo di lavoro e la libertà di sviluppo concessa: ho ritenuto importante la possibilità di elaborare personalmente un'architettura del caso d'uso con una visione ad alto livello, anziché il semplice sviluppo di un software predeterminato e dal percorso strettamente imposto.

\subsection{Obiettivi personali}
L'obiettivo fondamentale dello \stage\ è colmare il divario tra il mondo accademico e quello lavorativo.
Grazie al percorso di \stage\ in una ditta esterna ho avuto l'opportunità di conoscere l'ambiente di lavoro di un'azienda nel campo \sacr{ict}, facilitandomi l'inserimento nel mondo del lavoro.

Un altro obiettivo è ottenere una formazione riguardo la tecnologia di Kafka, che ritengo possa arricchire fortemente le mie capacità e \textit{skill} professionali.
Sono pertanto interessato a sviluppare la mia formazione riguardo l'utilizzo e le implicazioni di questa tecnologia in rapida espansione, la cui formazione potrà essermi utile in molti campi anche al di fuori degli obiettivi dell'azienda ospitante lo \stage.


\section{Il percorso di Stage}


% Descrizione di come il percorso di \textit{stage} si inserisce nella visione più ampia riportata qui sopra.
%
% \bigskip\noindent
% Elenco degli obiettivi del percorso:
% \begin{itemize}
%   \item formazione riguardo Kafka e l’ambito dell’integrazione;
%   \item verificare le capacità di Kafka nell’ambito EAI;
%   \item sperimentare l’utilizzo di Kafka come \textit{Middleware} tramite una simulazione di un caso d’uso reale a servizi indipendenti.
% \end{itemize}
%
% \noindent
% Breve esposizione dei motivi che mi hanno portato a scegliere questo percorso

% TODO:
% - motivazioni riguardo il percorso di stage proposto
% - descrizione del percorso di stage tenendo conto del contesto e degli obiettivi citati sopra
% - descrivere i miei obiettivi rispetto a quelli aziendali
\subsection{Obiettivi dello \textit{stage}}
\label{sub:obiettivi_stage}
Il percorso di \textit{stage} offerto dall'azienda si inserisce all'interno della strategia aziendale più ampia descritta sopra.
Al fine di esplorare la tecnologia di Apache Kafka nell'ambito di un \textit{Middleware} per l'integrazione aziendale, l'azienda ha proposto un percorso di \textbf{\textit{stage} il cui obiettivo è la re-ingegnerizzazione di un flusso di dati asincrono, utilizzando un'architettura basata su Kafka all'interno di un caso d'uso simulato tramite servizi indipendenti.}

Lo stagista ha il compito di osservare, testare e verificare che il \software\ possa svolgere alcuni compiti inerenti all'area del \sacr{eai}, analizzando alcuni casi d'uso presenti in un \textit{Middleware} aziendale in ambito \textit{telco}.
Il percorso di prevede una durata di 300 ore lavorative.

% \bigskip\noindent
% Il percorso proposto prevede le seguenti attività e obiettivi generali:
% \begin{itemize}
%   \item formazione riguardo la piattaforma di \textit{event streaming} Kafka e l'ambito dell'integrazione aziendale;
%   \item verifica delle capacità di Kafka nell'EAI;
%   \item sperimentazione e sviluppo di un \textit{Middleware} basato su Kafka tramite una simulazione di un caso d'uso reale a servizi indipendenti.
% \end{itemize}

\subsection{Prodotti attesi}

\begin{figure}[h]
  \begin{center}
    \includegraphics[width=0.8\textwidth]{images/kafka.png}
    \caption{Illustrazione di un sistema a servizi con Kafka}
    \captionsetup{aboveskip=2pt}
    \caption*{\begin{footnotesize}\textit{Fonte:} \url{https://kafka.apache.org/20/documentation.html}\end{footnotesize}}
  \end{center}
\end{figure}

I prodotti attesi al termine dello \textit{stage} sono dunque associati alla realizzazione di tre flussi di integrazione, basati su dei casi d’uso reali, per la gestione dei paradigmi di integrazione asincrono e asincrono con \textit{callback} (due requisiti obbligatori), e sincrono ove fosse disponibile del tempo aggiuntivo e se ritenuto opportuno; durante il percorso considerato il contesto e le opportunità offerte dal \software\ questo obiettivo verrà sostituito per testare delle funzionalità aggiuntive di Kafka.

Il prodotto \software\ finale sarà un sistema basato su servizi indipendenti costruito con un'architettura di tipo \sacr{eda} tramite l'utilizzo di Kafka (figura \thefigure).

\subsection{Contenuti formativi previsti}

La realizzazione di questi prodotti necessita una sostanziale formazione dello stagista riguardo i principali concetti del settore del \textit{Enterprise Application Integration} e l'utilizzo della piattaforma di \textit{event streaming} Kafka.
Più precisamente, i contenuti formativi previsti durante questo percorso di \textit{stage} sono i seguenti:
\begin{itemize}
  \item Concetti chiave del \gls{g_eai};
  \item \textit{Design architetturali};
  \item Cenni di \textit{Networking} applicato alle architetture distribuite;
  \item Architetture di Integrazione e \textit{Middleware};
  \item Apache Kafka.
\end{itemize}

\subsection{Interazione tra studente e referenti aziendali}
% Personalizzare definendo le modalità di interazione col tutor aziendale
Regolarmente, (almeno una volta la settimana) ci saranno incontri online (tramite la piattaforma Google Meet) con il tutor aziendale Francesco Giovanni Sanges, il responsabile dell’area \sacr{eai} Salvatore Dore e gli esperti delle tecnologie affrontate.
I meeting saranno necessariamente online, dato il dislocamento dei vari membri in diverse città.

Lo scopo di questi incontri sarà quello di verificare lo stato di avanzamento, chiarire gli obiettivi ove necessario, affinare la ricerca e aggiornare la pianificazione iniziale.

% \subsection{Strumenti organizzativi dello \textit{stage}}



\subsection{\textit{Way of working}}

% Descrizione del Piano di Lavoro presentato, le strategie e il metodo di lavoro stabilito.
All'inizio del percorso ho delineato un \textit{way of working}, ovvero un metodo di lavoro da mantenere per tutta la durata dello \textit{stage}, insieme al tutor aziendale, il responsabile dell'area \sacr{eai} e gli esperti del settore.

Per garantire un buon livello organizzativo, quantificare l'avanzamento e rendere agevole la verifica e il supporto tecnico l'azienda ha proposto l'utilizzo di una \textit{board} di progetto.
% Tra le tante opzioni disponibili, mi è stata proposta la piattaforma \textit{ClickUp}.
% Rispetto alla concorrenza questa \textit{board} è ricca di funzionalità, pulita nell'esposizione dello stato del progetto, e la maggior parte delle sue funzioni sono gratuite.

L'organizzazione efficiente del progetto è dunque garantita dall'utilizzo dei vari strumenti a supporto, quali \textit{Kanban Board} (come \textit{Click Up}\footfullcite{clickup} per la gestione di progetto e \textit{Notion}\footfullcite{notion} per le prenotazioni della postazione di lavoro in sede), \textit{chat} (come \textit{Google Chat}\footfullcite{google-chat}) per i confronti rapidi con gli altri membri interni al progetto ed e-mail per le comunicazioni con componenti esterni al progetto.

Lo strumento più utilizzato in ambito organizzativo durante il mio percorso è la \textit{Kanban Board} di \textit{Click Up}, che ha permesso la gestione, il confronto, la quantificazione e la verifica del progresso.
Tra le tante opzioni disponibili \textit{ClickUp} possiede numerosi vantaggi rispetto la concorrenza: la piattaforma è ricca di funzionalità, pulita nell'esposizione dello stato del progetto, e la maggior parte delle sue funzioni sono gratuite.
La figura seguente illustra, a titolo esemplificativo, uno \textit{screenshot} che raffigura lo stato dell'avanzamento.

% \bigskip
\begin{figure}[h]
  \begin{center}
    \includegraphics[width=\textwidth]{images/clickup_board_v2.png}
    \caption{\textit{Kanban Board} del progetto di \textit{stage}}
    \captionsetup{aboveskip=2pt}
    \caption*{\begin{footnotesize}\textit{Fonte: elaborazione personale}\end{footnotesize}}
  \end{center}
\end{figure}

Le attività (\textit{task}) vengono inizialmente create nella colonna "\textsc{DA FARE}" dal tutor aziendale o dal sottoscritto, ove ritenuto opportuno.
Per dimostrare l'avanzamento il \textit{task} si sposta verso destra a seconda dello stato raggiunto; lo stagista ha la responsabilità del cambiamento di stato fino alla colonna "DA VERIFICARE", dopodiché è compito del tutor aziendale la verifica e lo spostamento del \textit{task} in "\textsc{TASK APPROVATI}", che comporta l'approvazione finale e conclusione dell'attività.

Per tenere traccia del lavoro svolto riguardante una specifica attività ho utilizzato le \textit{card} messe a disposizione dalla piattaforma, che mi hanno consentito di delineare precisamente la pianificazione e descrizione dell'avanzamento in dettaglio del singolo \textit{task}.

\begin{figure}[H]
  \begin{center}
    \includegraphics[width=\textwidth]{images/clickup_task_v2.png}
    \caption{Esempio di un'attività del processo di Formazione}
    \captionsetup{aboveskip=2pt}
    \caption*{\begin{footnotesize}\textit{Fonte: elaborazione personale}\end{footnotesize}}
  \end{center}
\end{figure}

Questa \textit{card} contiene una casella di testo per inserire una descrizione e appunti utili ove sia richiesto, una \textit{checklist} approfondita, e una colonna che mantiene uno storico dei commenti; quest'ultima colonna non solo permette a me di mantenere un'importante resoconto sul lavoro svolto, ma consente anche al tutor aziendale e esperti del settore di quantificare il progresso e di fornire un aiuto rapido e contestuale.

Per la condivisione di codice è stato possibile utilizzare uno strumento a mia discrezione, e pertanto ho utilizzato Git per creare una \textit{repository} e successivamente ne ho fatto l'\textit{upload} in modalità privata (secondo indicazioni aziendali) su Github.
L'inserimento del codice identificativo del \texttt{commit} all'interno di un commento allo scopo di \textit{log} ha favorito ulteriormente il tracciamento del progresso e reso agevole l'eventuale supporto da parte degli esperti aziendali.

All'inizio del percorso il tutor aziendale e il responsabile del \sacr{eai} hanno creato delle \textit{card} contenenti le attività previste per ogni settimana (\textit{task}) al fine di fornire una struttura generale del progetto.
All'interno di questi \textit{task} vi sono i concetti chiave, attività previste e obiettivi settimanali che lo stagista è tenuto a seguire per garantire l'efficacia del prodotto finale.
Oltre a questi \textit{task} principali, ho potuto creare di \textit{task} ausiliari e dei \textit{sub-task} per descrivere più adeguatamente l'attività in corso.

Ciascun \textit{task} contiene una colonna laterale dove ho mantenuto un \textit{log} di tutto ciò che è stato eseguito relativo al \textit{task} in questione, allo scopo di esplicitarne il progresso e rendere agevole un eventuale supporto dal tutor o l'evoluzione futura.

Ogni settimana è previsto un \textit{online meeting} per la verifica del progresso ove necessario, la risposta ad eventuali questioni sollevate, e spiegazioni riguardo lo sviluppo della settimana successiva.
Alcune di queste videoconferenze ha visto la partecipazione di altri esperti che mi hanno aiutato a comprendere meglio il caso d'uso da re-ingegnerizzare, riassumendo lo stato attuale del sistema d'integrazione per uno dei clienti con relativi \textit{file} utilizzati.

Per mantenere alto il livello di organizzazione, efficienza ed efficacia, all'inizio di ogni giornata lavorativa ho creato un breve piano giornaliero con successivo consuntivo a fine giornata.
Questo ha permesso al tutor di verificare rapidamente il corretto avanzamento del processo in corso e a me di mantenere il \textit{focus} su di esso.


\subsection{Pianificazione del lavoro}
\label{sec:pianificazione}
Ad ogni incremento è associato un requisito obbligatorio, desiderabile o facoltativo.
A questi requisiti vi è associato un codice identificativo per favorirne il tracciamento futuro, in che precede la voce descrittiva dell'incremento.
\noindent
Ogni codice è composto da una lettera seguita da dei numeri interi, secondo il seguente modello:
\begin{center}
	\textbf{A-X.Y.Z}
\end{center}
ove, da sinistra verso destra:
\begin{itemize}

  \item \textbf{A} rappresenta la lettera che qualifica il requisito come obbligatorio, desiderabile o facoltativo, secondo la seguente notazione:
  \begin{itemize}
  	\item \textit{O} per i requisiti obbligatori, vincolanti in quanto obiettivo primario richiesto dal committente;
  	\item \textit{D} per i requisiti desiderabili, non vincolanti o strettamente necessari,
  		  ma dal riconoscibile valore aggiunto;
  	\item \textit{F} per i requisiti facoltativi, rappresentanti valore aggiunto non strettamente
  		  competitivo.
  \end{itemize}

  \item \textbf{X} rappresenta la settimana in cui viene inizialmente pianificato l'incremento (identificata da un numero incrementale e intero, partendo da 1).
  Questo consente allo studente, al tutor interno e al tutor interno una rapida quantificazione dell'avanzamento corrente dello stage rispetto a quanto inizialmente pianificato.

  \item \textbf{Y} rappresenta la posizione sequenziale prevista dell’incremento all’interno della settimana (incrementale e intero, partendo da 1). Esso è strettamente associato alla lettera.


\end{itemize}
\noindent
Di seguito viene presentata la pianificazione settimanale delle ore lavorative previste.
Ad ogni settimana sono assegnate le voci contenenti gli incrementi previsti in essa, ove i codici utilizzano la notazione descritta precedentemente.

\noindent
Tutte le settimane prevedono 40 ore lavorative, fatta eccezione per l'ultima che ne prevede 20.

% \usepackage{lscape}
% \newcolumntype{b}{>{\hsize=0.2\textwidth}X}
% \newcolumntype{s}{>{\arraybackslash\hsize=0.2\textwidth}X}
% \newcolumntype{m}{>{\arraybackslash\hsize=0.6\textwidth}X}
\onehalfspacing
\begin{small}
  \begin{center}
    \centering
    \renewcommand\arraystretch{1.6}
    \begin{longtable}{| >{\centering\arraybackslash}m{2cm}|m{1.2cm}|m{10.5cm}|}
      \hline
      \textsc{\textbf{Settimana}} & \textsc{\textbf{Codice}} & \textsc{\textbf{Task associati}} \\
      \hline

      \multirow{8}{*}{\normalsize\textbf{1}}
      & \centering O-1.1 & Incontro con le persone coinvolte nel progetto per discutere i requisiti e le richieste relative al sistema da sviluppare\\
      \cline{2-3}
      & \centering O-1.2 & Verifica credenziali e strumenti di lavoro assegnati\\
      \cline{2-3}
      & \centering O-1.3 & Presa visione dell’infrastruttura esistente\\
      \cline{2-3}
      & \centering D-1.1 & Ripasso approfondito riguardo i seguenti argomenti:
      \smallskip
        \begin{itemize}
           \item Ingegneria del \software;
           \item Sistemi di versionamento;
           \item Architetture \software;
           \item Cenni di \textit{Networking}.
         \end{itemize} \\
     \Xhline{2\arrayrulewidth}

      \multirow{8}{*}{\normalsize\textbf{2}}
      & \centering O-2.1 & Nozioni fondamentali riguardo \sacr{eai} e \sacrfoot{soa}\\
      \cline{2-3}
      & \centering O-2.2 & Approfondimenti riguardo le Architetture a Messaggio, in particolare:
        \begin{itemize}
           \item \textit{Integration Styles};
           \item \textit{Channel Patterns};
           \item \textit{Message Construction Patterns};
           \item \textit{Routing Patterns};
           \item \textit{Transformation Patterns};
           \item \textit{System Management Patterns}.
         \end{itemize} \\
     \Xhline{2\arrayrulewidth}

      \multirow{2}{*}{\normalsize\textbf{3}}
      & \centering O-3.1 & Apache Kafka:
        \begin{itemize}
            \item Introduzione a Kafka;
            \item Concetti fondamentali di Kafka;
            \item Avvio e \sacrfoot{cli};
            \item Programmazione in Kafka con Java.
          \end{itemize} \\
      \cline{2-3}
      & \centering D-3.1 & Esempi e applicazioni di Apache Kafka \\
     \Xhline{2\arrayrulewidth}

     \multirow{1}{*}{\normalsize\textbf{4}}
     & \centering O-4.1 & Confluent Platform:
       \begin{itemize}
         \item \textit{Service registry};
         \item \sacr{rest} \textit{proxy};
         \item kSQL;
         \item Confluent \textit{connectors};
         \item \textit{Control center}.
       \end{itemize} \\
    \Xhline{2\arrayrulewidth}

    \multirow{2}{*}{\normalsize\textbf{5}}
    & \centering O-5.1 & Analisi dei casi d'uso reali\\
    \cline{2-3}
    & \centering O-5.2 & Realizzazione dei componenti per l'esecuzione dei casi di test\\
    \Xhline{2\arrayrulewidth}

    \multirow{1}{*}{\normalsize\textbf{6}}
    & \centering O-6.1 & Analisi re-ingegnerizzazione e collaudo del flusso di integrazione asincrono\\
    \Xhline{2\arrayrulewidth}

    \multirow{1}{*}{\normalsize\textbf{7}}
    & \centering O-7.1 & Analisi e re-ingegnerizzazione e collaudo del flusso di integrazione asincrono con callback\\
    \Xhline{2\arrayrulewidth}

    \multirow{1}{*}{\normalsize\textbf{8}}
    & \centering O-8.1 & Analisi e re-ingegnerizzazione e collaudo del flusso di integrazione sincrono\\
    \hline
    \caption{Pianificazione settimanale dello \textit{stage}}
    \label{tab:pianificazione}
    \end{longtable}
  \end{center}
\end{small}
% \doublespacing

% \begin{table}
%  \begin{tabularx}{\textwidth}{X*{5}{>{\centering\arraybackslash}X}}
%  \hline
%  Sample window & Probability cutoff value & Crises Correctly called (\%) &
%  Non-crises correctly called (\%) & Missed crises & False Alarm  \\
%  \hline
% \end{tabular}
% \end{table}
%
% \begin{itemize}
%        \item \textbf{Prima Settimana (40 ore)}
%        \begin{itemize}
%            \item \textbf{O-1.1} Incontro con le persone coinvolte nel progetto per discutere i requisiti e le richieste relative al sistema da sviluppare;
%            \item \textbf{O-1.2} Verifica credenziali e strumenti di lavoro assegnati;
%            \item \textbf{O-1.3} Presa visione dell’infrastruttura esistente;
%            \item \textbf{D-1.1} Ripasso approfondito riguardo i seguenti argomenti:
%              \begin{itemize}
%                \item Ingegneria del \software;
%                \item Sistemi di versionamento;
%                \item Architetture \software;
%                \item Cenni di \textit{Networking}.
%              \end{itemize}
%        \end{itemize}
%
%        \item \textbf{Seconda Settimana (40 ore)}
%        \begin{itemize}
%            \item \textbf{O-2.1} Nozioni fondamentali riguardo \sacr{eai} e \sacrfoot{soa};
%            \item \textbf{O-2.2} Approfondimenti riguardo le Architetture a Messaggio, in particolare:
%              \begin{itemize}
%                \item \textit{Integration Styles};
%                \item \textit{Channel Patterns};
%                \item \textit{Message Construction Patterns};
%                \item \textit{Routing Patterns};
%                \item \textit{Transformation Patterns};
%                \item \textit{System Management Patterns}.
%              \end{itemize}
%        \end{itemize}
%
%        \item \textbf{Terza Settimana (40 ore)}
%        \begin{itemize}
%            \item \textbf{O-3.1} Apache Kafka:
%              \begin{itemize}
%                \item Introduzione a Kafka;
%                \item Concetti fondamentali di Kafka;
%                \item Avvio e \sacrfoot{cli};
%                \item Programmazione in Kafka con Java.
%              \end{itemize}setup, aboveskip, footnotesize, png, textwidth, esb, soa, jpeg, jpg, Kanban, bigskip,
%            \item \textbf{D-3.1} Esempi e applicazioni di Apache Kafka.
%        \end{itemize}
%
%
%        \item \textbf{Quarta Settimana (40 ore)}
%        \begin{itemize}
%            \item \textbf{O-4.1} Confluent Platform:
%              \begin{itemize}
%                \item Service registry;
%                \item REST proxy;
%                \item kSQL;
%                \item Confluent connectors;
%                \item Control center.
%              \end{itemize}
%        \end{itemize}
%
%        \item \textbf{Quinta Settimana (40 ore)}
%        \begin{itemize}
%          \item \textbf{O-5.1} Analisi dei casi d'uso reali;
%          \item \textbf{O-5.2} Realizzazione dei componenti per l'esecuzione dei casi di test.
%        \end{itemize}
%
%        \item \textbf{Sesta Settimana (40 ore)}
%        \begin{itemize}
%          \item \textbf{O-6.1} Analisi reingegnerizzazione e collaudo del flusso di integrazione asincrono.
%        \end{itemize}
%
%        \item \textbf{Settima Settimana (40 ore)}
%        \begin{itemize}
%            \item \textbf{O-7.1} Analisi e reingegnerizzazione e collaudo del flusso di integrazione asincrono con callback
%
%        \end{itemize}
%
%        \item \textbf{Ottava Settimana (20 ore)}
%        \begin{itemize}
%            \item \textbf{D-8.1} Analisi e reingegnerizzazione e collaudo del flusso di integrazione sincrono.
%        \end{itemize}
%
% \end{itemize}

\begin{figure}[h]
  \begin{center}
    \includegraphics[width=\textwidth]{images/gantt.png}
    \caption{Diagramma di Gantt del piano di lavoro}
    \captionsetup{aboveskip=2pt}
    \caption*{\begin{footnotesize}\textit{Fonte: elaborazione personale}\end{footnotesize}}
  \end{center}
\end{figure}

\noindent
Secondo questa pianificazione, (di cui la figura \thefigure\ rappresenta il diagramma di Gantt) le 300 ore di \stage\ previste sono approssimativamente divise in:
\begin{itemize}
  \item 160 ore di Formazione sulle tecnologie;
  \item 60 ore di Progettazione dei componenti e dei test;
  \item 60 ore di Sviluppo dei componenti e dei test;
  \item 20 ore di Valutazioni finali, Collaudo e Presentazione della Demo.
\end{itemize}
