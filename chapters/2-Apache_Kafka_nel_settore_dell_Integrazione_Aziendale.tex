\chapter{Apache Kafka nel settore dell’Integrazione Aziendale}

\section{L’evoluzione delle architetture di integrazione}

Introduzione al motivo aziendale per cui è nato questo percorso di stage: la propensione odierna alle architetture a microservizi, la gestione di grandi flussi di dati in modo efficiente ed efficace, una richiesta di un sistema innovativo da parte della clientela.

\bigskip\noindent
Esposizioni delle ragioni personali che hanno portato alla scelta di tale percorso.


\section{Apache Kafka come Middleware}

Introduzione ad Apache Kafka: la principale piattaforma di event streaming.

\bigskip\noindent
Come Kafka possa risolvere i problemi e le necessità viste qui sopra.

\section{Il percorso di Stage}

Descrizione di come il percorso di \textit{stage} si inserisce nella visione più ampia riportata qui sopra.

\bigskip\noindent
Elenco degli obiettivi del percorso:
\begin{itemize}
  \item formazione riguardo Kafka e l’ambito dell’integrazione;
  \item verificare le capacità di Kafka nell’ambito EAI;
  \item sperimentare l’utilizzo di Kafka come \textit{Middleware} tramite una simulazione di un caso d’uso reale a servizi indipendenti.
\end{itemize}

\noindent
Breve esposizione dei motivi che mi hanno portato a scegliere questo percorso
