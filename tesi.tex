\documentclass[a4paper]{report}

\usepackage[T1]{fontenc}
\usepackage[utf8]{inputenc}
% \usepackage{fancyhdr}

%Lingua
\usepackage[italian]{babel}
\usepackage{setspace}

% %Pacchetto per definire layout di pagina
\usepackage[left=3cm, right=3cm, top=3cm, bottom=3cm]{geometry}


%Inclusione immagini
\usepackage{graphicx}
%Colori
\usepackage[usenames]{color}
\usepackage{xcolor}
%Crea link ipertestuali
\usepackage{hyperref}
% \usepackage{xurl}
% \usepackage{ragged2e}
% \justifying


%Formattazione url
% \usepackage{url}\
%
% \usepackage{csquotes}
% \usepackage[backend=biber, hyperref=true]{biblatex}
% \addbibresource{tesi.bib}
% \DeclareFieldFormat{formaturl}{#1}
% \newbibmacro*{url+urldate}{%
% \printtext[formaturl]{%
%   \iffieldundef{urlyear}
%     {}
%     {\setunit*{\addspace}%
%      Visitato il
%      \printtext[urldate]{\printurldate}}}
%    \newline
%    \printfield{url}}%

\urlstyle{tt}

\begin{document}

% \newcommand{\textgreek}[1]{\begingroup\fontencoding{LGR}\selectfont#1\endgroup}
% \newcommand\frontmatter{%
%     \cleardoublepage
%   %\@mainmatterfalse
%   \pagenumbering{roman}}
%
% \newcommand\mainmatter{%
%     \cleardoublepage
%  % \@mainmattertrue
%   \pagenumbering{arabic}}

% \pagestyle{fancy}
% \fancyhead{}

\renewcommand{\chaptermark}[1]{\markboth{\small\textsc{\thechapter.\ #1}}{}}
\renewcommand{\sectionmark}[1]{\markright{\small\textsc{\thesection.\ #1}}{}}
% \fancyhead[LE]{\rightmark}
% \fancyhead[RO]{\leftmark}


% \frontmatter

\begin{titlepage}
\begin{center}
\includegraphics[scale=0.1]{images/logo.png}\\

%Per il frontespizio del dipartimenti di Ing. dell'Informazione commentare le riga precedente e decommentare la successiva
%\includegraphics[scale=0.2]{images/logo_unipd.png} \hfill \includegraphics[scale=0.2]{images/logo_dei.png}\\
\vspace{0.8cm}
\textsc{\LARGE Università degli Studi di Padova}\\
\vspace{0.45cm}
\textsc{\large Dipartimento di Matematica}\\
\vspace{0.45cm}
\textsc{\large Corso di Laurea in}\\
\textsc{\large Informatica}\\
\vspace{0.45cm}
\textsc{\large Elaborato finale}\\
\vfill
% Title
{ \LARGE \bfseries Sperimentazione di Apache Kafka per l'integrazione funzionale di un'applicazione aziendale}\\
\vspace{0.45cm}
{ \large \bfseries Experimenting with Apache Kafka for the Integration of an Enterprise Application}\\
\vfill
\textit{Relatore:} \hfill \textit{Laureando:}\\
\textsc{Prof. Tullio Vardanega} \hfill \textsc{Andrea Dorigo}\\
\hfill \textsc{1170610}\\

\vfill
% Bottom of the page
{\large Anno Accademico 2020/2021}
\end{center}
\end{titlepage}

\thispagestyle{empty} %pagina bianca dopo il titolo
\cleardoublepage
%
% \pagenumbering{roman} %numerazione romana per l'indice, l'abstract e i ringraziamenti
% \thispagestyle{empty}
%
% \clearpage{\pagestyle{plain}\cleardoublepage}
% \input{abstract.tex}
%
% \clearpage{\pagestyle{plain}\cleardoublepage}
\doublespacing
\tableofcontents{} %Indice

% \mainmatter
\chapter{Contesto aziendale}

\section{Dominio applicativo}
% Breve introduzione al settore del \textit{Enterprise Application Integration}, al tipo di clientela (ovvero pubblica e privata di grandi dimensioni, big data), alla tipologia di \textit{software} prodotti dall’azienda per la clientela (\textit{Middleware}), e alla propensione all’innovazione (richieste da parte della clientela
In conclusione al percorso di studi del corso di laurea in Informatica ho effettuato lo \textit{stage} presso \textit{Sync Lab}.
È un'azienda di produzione \textit{software} e integrazione di sistemi che fornisce principalmente prodotti a clienti di grande dimensione, sia pubblici che privati.

L'azienda è suddivisa in molteplici settori con diverse sedi; l'esperienza personale mi ha portato a conoscere il settore dell'\textit{Enterprise Architecture Integration} e del \textit{Tecnical Professional Services Padova}.
Il percorso di \textit{stage} intrapreso è associato al primo di questi, che si occupa principalmente dell'EAI (\textit{Enterprise Application Integration}) ovvero dell'integrazione funzionale di applicazioni aziendali per una clientela di grandi dimensioni, tramite sistemi di integrazione \textit{Middleware}.
Il caso d'uso realizzato nel mio percorso ha simulato un grande cliente gestore di telecomunicazioni, secondo una visione coerente con il tipo di clientela reale dell'azienda.

I \textit{Middleware} prodotti comprendono l'utilizzo di molteplici linguaggi e tecnologie in continua evoluzione.
È un contesto con un'importante propensione all'innovazione, talvolta esplicitamente richiesta dai clienti: l'evoluzione attuale riguarda la migrazione verso sistemi sempre più distribuiti, in grado di gestire efficacemente flussi di dati in continua crescita.

\section{L’evoluzione delle architetture di integrazione}

% Introduzione al motivo aziendale per cui è nato questo percorso di stage: la propensione odierna alle architetture a microservizi, la gestione di grandi flussi di dati in modo efficiente ed eff1icace, una richiesta di un sistema innovativo da parte della clientela.3
L'evoluzione del settore dell'\textit{Enterprise Application Integration} verso soluzioni sempre più distribuite e con un flusso di dati in continuo aumento ha sviluppato nei clienti (e di conseguenza nell'azienda) un interesse verso il prodotto Apache Kafka.
Il software ha dimostrato negli anni recenti un notevole successo in diversi campi; l'azienda ha interesse nel testare le capacità di Kafka nel soddisfare le esigenze dell'integrazione aziendale.

\bigskip
\begin{figure}[h]
  \begin{center}
    \includegraphics[width=0.8\textwidth]{images/kafka.png}\\
    \caption{Illustrazione di un sistema a servizi con Apache Kafka}
    \captionsetup{aboveskip=2pt,font=it}
    \caption*{Fonte: https://kafka.apache.org/20/documentation.html}
  \end{center}
\end{figure}

Kafka è una piattaforma di \textit{event streaming}, un sistema moderno e distribuito basato sugli eventi anzichè su di una soluzione più classica come può essere quella del \textit{request/response}.
L'adozione del \textit{software} nell'ambito dell'EAI è in crescita dato le dimostrate qualità nel gestire grandi moli di dati: la sua performance, sicurezza e scalabilità sono i punti che hanno portato il software al suo attuale successo.

L'interesse aziendale nel \textit{software} risiede nell'utilizzo di Kafka come un  \textit{Middleware}, ovvero di un sistema che consenta la comunicazione tra differenti servizi con un rapido flusso di dati fra di essi.
L'azienda ha avviato un percorso per testare le capacità di Kafka rispetto agli attuali strumenti utilizzati nel settore, per valutare i vantaggi e svantaggi che l'adozione di tale software può fornire al cliente.

\bigskip\noindent
Sono numerosi i vantaggi che Kafka può portare nel settore, fra cui:
\begin{itemize}
  \item gestione performante di un enorme flusso di dati;
  \item scalabilità;
  \item sicurezza riguardo la persistenza dei dati;
  \item semplice integrazione e affiancamento a sistemi già esistenti;
  \item l'essere una piattaforma \textit{open source};
  \item processazione dei dati integrata.
\end{itemize}


% \bigskip\noindent
% Esposizioni delle ragioni personali che hanno portato alla scelta di tale percorso.


\section{Processi interni e strumenti organizzativi}

% Esposizione delle norme organizzative (\textit{online meeting}, \textit{smart working}, presenze in sede), degli strumenti utilizzati nel rapporto con l’azienda (chat, email e \textit{Project Board}), e delle norme di progetto.
%
% \bigskip\noindent
% Processi interni in cui sono stato coinvolto: Sviluppo, Collaudo, Verifica, Formazione, Manutenzione/Evoluzione.

% \bigskip\noindent
% Breve presentazione dei ruoli delle persone coinvolte nel percorso di \textit{stage}.

L'azienda adotta dei processi interni per delineare l'avanzamento di un progetto.
Durante il percorso di \textit{stage} sono stato coinvolto nei processi di Formazione, Sviluppo, Verifica e Collaudo; i processi di Manutenzione ed Evoluzione sono stati solamente accennati in quanto al di fuori dello scopo del percorso.
Questi processi, nella mia esperienza personale, non sono stati delineati rigorosamente, al fine di garantire una certa rapidità e adattabilità al progetto di sperimentazione.
Ogni processo è suddiviso in attività modulari, per rendere l'avanzamento efficace e quantificabile.

L'organizzazione efficiente di un progetto è garantita dall'utilizzo dei vari strumenti a supporto, quali \textit{Kanban Board} (come \textit{Click Up} per la gestione di progetto e \textit{Notion} per le prenotazioni della postazione di lavoro in sede), \textit{chat} (come \textit{Google Chat}) per i confronti rapidi con gli altri membri interni al progetto ed e-mail per le comunicazioni con componenti esterni al progetto.

Lo strumento più utilizzato in ambito organizzativo durante il percorso è la \textit{Kanban Board} di \textit{Click Up}, che ha permesso la gestione, il confronto, la quantificazione e la verifica del progresso.
Le figure seguenti illustrano alcuni \textit{screenshot} che raffigurano lo stato dell'avanzamento.

\bigskip
\begin{figure}[h]
  \includegraphics[width=\textwidth]{images/clickup_board_v2.png}\\
  \caption{\textit{Kanban Board} del progetto di \textit{stage}}
\end{figure}

Le attività (\textit{task}) vengono inizialmente create nella colonna "DA FARE" dal tutor aziendale o da me, ove ritenuto opportuno.
Per dimostrare l'avanzamento il \textit{task} si sposta verso destra a seconda dello stato raggiunto; lo stagista ha la responsabilità del cambiamento di stato fino alla colonna "DA VERIFICARE", dopodichè è compito del tutor aziendale la verifica e lo spostamento del \textit{task} in "TASK APPROVATI", che comporta l'approvazione finale e conclusione dell'attività.

Per tenere traccia del lavoro svolto riguardante una specifica attività ho utilizzato le \textit{card} messe a disposizione dalla piattaforma, che mi hanno consentito di delineare precisamente la pianificazione e descrizione dell'avanzamento in dettaglio del singolo \textit{task}.
Questa \textit{card} contiene una casella di testo per inserire una descrizione e appunti utili ove sia richiesto, una \textit{checklist} approfondita, e una colonna che mantiene uno storico dei commenti; quest'ultima colonna non solo permette a me di mantenere un'importante resoconto sul lavoro svolto, ma consente anche al tutor aziendale e esperti del settore di quantificare il progresso e di fornire un aiuto rapido e contestuale.

\begin{figure}[H]
  \includegraphics[width=\textwidth]{images/clickup_task_v2.png}\\
  \caption{Esempio di un'attività del processo di Formazione}
\end{figure}


\section{Ambiente di lavoro}

% Sviluppo indipendente dal sistema operativo, produzione di \textit{software} non strettamente legati ad uno specifico linguaggio, utilizzo di ambienti virtuali quali \textit{Virtual Machine} e \textit{container} per simulare sistemi indipendenti.
L'ambiente di lavoro di cui ho avuto esperienza risulta libero e flessibile.
Lo sviluppo del prodotto nell'ambito del EAI dev'essere indipendente dal linguaggio di programmazione, dagli strumenti utilizzati per l'esecuzione e sviluppo, e possibilmente anche dal Sistema Operativo su cui eseguire il \textit{software}.
A tal scopo si utilizzano strumenti quali \textit{Virtual Machine} e \textit{Container}: essi non solo garantiscono l'indipendenza dal Sistema Operativo in uso, ma simulano efficacemente il caso d'uso reale in cui i vari eseguibili sono dislocati in più computer o server come spesso accade per il cliente.

Nonostante il percorso formativo abbia visto l'apprendimento di entrambe le tecnologie tramite l'utilizzo dei \textit{software} \textit{Virtual Box} e \textit{Docker}, solo quest'ultima è stata utilizzata durante il progetto poichè più efficente e minimale.
Più precisamente, ho utilizzato l'estensione \textit{Docker-compose} per gestire in modo elegante la generazione e collaudo di più servizi indipententi: non solo questo \textit{software} consente di creare una rete di \textit{container} comunicanti, ma rende anche rapido ed efficaciente lo sviluppo grazie alla possibilità di modificare e riavviare un singolo servizio all'interno del sistema.

\chapter{Apache Kafka nell’Integrazione Aziendale}

% \section{L'evoluzione delle architetture di integrazione}
\section{Obiettivi aziendali}

Per soddisfare le richieste clientelari ed essere sempre competitiva e all'avanguardia, una priorità di Sync Lab sono le esplorazioni tecnologiche e di prodotto anche tramite l'utilizzo di percorsi di \stage\ insieme ai laureandi, come quanto accaduto nella mia esperienza.
Questi percorsi consentono all'azienda non solo di testare l'utilizzo di nuovi \software\ ma anche di conoscere e mettere alla prova le capacità del laureando in vista di una potenziale assunzione al termine dello \stage.
% Introduzione al motivo aziendale per cui è nato questo percorso di stage: la propensione odierna alle architetture a microservizi, la gestione di grandi flussi di dati in modo efficiente ed eff1icace, una richiesta di un sistema innovativo da parte della clientela.3
\begin{figure}[h]
  \begin{center}
    \includegraphics[width=0.6\textwidth]{images/distributed.png}
    \caption{Illustrazione di un sistema distribuito}
    \captionsetup{aboveskip=2pt}
    \caption*{\begin{footnotesize}\textit{Fonte:} \url{https://www.delphitools.info/DWSH/}\end{footnotesize}}
  \end{center}
\end{figure}

Nel settore dell'\gls{g_eai}, l'evoluzione tecnologica è diretta verso soluzioni sempre più distribuite e con un flusso di dati in continuo aumento.
Uno degli obiettivi specifici nell'area \sacr{eai} di Sync Lab è pertanto quello di trovare un \software\ o tecnologia in grado di soddisfare i bisogni dei clienti di gestire un flusso di dati di dimensioni molto maggiori a quelle attuali, tramite architetture a messaggio che utilizzano servizi distribuiti.

% \section{L’evoluzione delle architetture di integrazione}
%
% % Introduzione al motivo aziendale per cui è nato questo percorso di stage: la propensione odierna alle architetture a microservizi, la gestione di grandi flussi di dati in modo efficiente ed eff1icace, una richiesta di un sistema innovativo da parte della clientela.3
% L'evoluzione del settore dell'\textit{Enterprise Application Integration} verso soluzioni sempre più distribuite e con un flusso di dati in continuo aumento ha sviluppato nei clienti (e di conseguenza nell'azienda) un interesse verso il prodotto Apache Kafka.
% Il software ha dimostrato negli anni recenti un notevole successo in diversi campi; l'azienda ha interesse nel testare le capacità di Kafka nel soddisfare le esigenze dell'integrazione aziendale.
%
% \bigskip
% \begin{figure}[h]
%   \includegraphics[width=\textwidth]{images/kafka.png}\\
%   \caption{Illustrazione di un sistema a servizi con Apache Kafka}
%   \captionsetup{aboveskip=2pt,font=it}
%   \caption*{Fonte: https://kafka.apache.org/20/documentation.html}
% \end{figure}
%
% Kafka è una piattaforma di \textit{event streaming}, un sistema moderno e distribuito basato sugli eventi anzichè su di una soluzione più classica come può essere quella del \textit{request/response}.
% L'adozione del \software\ nell'ambito dell'EAI è in crescita dato le dimostrate qualità nel gestire grandi moli di dati: la sua performance, sicurezza e scalabilità sono i punti che hanno portato il software al suo attuale successo.
%
% % \bigskip\noindent
% % Esposizioni delle ragioni personali che hanno portato alla scelta di tale percorso.


\subsection{Kafka come Middleware}

Per soddisfare le esigenze di innovazione l'azienda ha avviato un percorso per indagare le capacità del software Apache Kafka nell'ambito dell'integrazione aziendale.

\bigskip
\begin{figure}[h]
  \begin{center}
    \includegraphics[width=0.14\textwidth]{images/kafka_logo.png}
    \caption{Logo di Apache Kafka}
    \captionsetup{aboveskip=2pt}
    \caption*{\begin{footnotesize}\textit{Fonte:} \url{https://commons.wikimedia.org/wiki/File:Apache_kafka.svg}\end{footnotesize}}
  \end{center}
\end{figure}

Kafka è una piattaforma di \textit{event streaming}, un sistema distribuito e moderno basato sugli eventi anzichè su di una soluzione più classica come può essere quella del \textit{request/response}.
Apache Kafka si integra ottimamente in molti sistemi basati sulle Architetture a messaggio, in cui lo scambio affidabile di dati in tempo reale è essenziale.

Il \software\ ha dimostrato negli anni recenti un notevole successo in diversi campi\footnote{Fonte: \url{https://kafka.apache.org/powered-by}}, come quello del flusso di \textit{Big Data}, del monitoraggio e dell'elaborazione dati in tempo reale.
L'adozione del \software\ nell'ambito dell'EAI è in crescita dato le dimostrate qualità nel gestire grandi moli di dati: la sua performance, sicurezza e scalabilità sono i punti che hanno portato il software al suo attuale successo.

L'interesse di Sync Lab nel \software\ risiede dunque nell'utilizzo di Kafka come un  \textit{Middleware} per soddisfare i problemi di integrazione aziendale e reingegnerizzare i flussi di dati preesistenti, ovvero sviluppare un nuovo sistema che consenta la comunicazione tra differenti servizi con un rapido flusso di dati fra di essi.


\bigskip
\begin{figure}[h]
  \begin{center}
    \includegraphics[width=0.8\textwidth]{images/kafka.png}
    \caption{Illustrazione di un sistema a servizi con Kafka}
    \captionsetup{aboveskip=2pt}
    \caption*{\begin{footnotesize}\textit{Fonte:} \url{https://kafka.apache.org/20/documentation.html}\end{footnotesize}}
  \end{center}
\end{figure}


L'azienda ha avviato un percorso per testare le capacità di Kafka rispetto agli attuali strumenti utilizzati nel settore, per valutare i vantaggi e svantaggi che l'adozione di tale \software\ può fornire al cliente.

\bigskip\noindent
Sono numerosi i vantaggi che Kafka può portare nel settore, fra cui:
\begin{itemize}
  \item gestione rapida e performante di un enorme flusso di dati;
  \item scalabilità;
  \item sicurezza riguardo la persistenza dei dati;
  \item semplice integrazione e affiancamento a sistemi già esistenti;
  \item l'essere una piattaforma \textit{open source};
  \item processazione dei dati in tempo reale integrata.
\end{itemize}

% \bigskip\noindent
% TODO: Come Kafka possa risolvere i problemi e le necessità viste qui sopra.

\section{Motivazioni e obiettivi personali}

\subsection{Scelta del percorso}

Una delle ragioni che mi ha portato a scegliere questo percorso di \stage\ è l'interesse verso Apache Kafka.
L'utilizzo della piattaforma di \textit{event streaming} è sempre più in crescita, come l'evoluzione verso sistemi sempre più distribuiti e a microservizi.

Un altro fattore fondamentale alla scelta del percorso sono stati la famigliarità con l'azienda, il personale giudizio positivo che ho avuto riguardo il loro metodo di lavoro e la libertà di sviluppo concessa: ho ritenuto importante la possibilità di elaborare personalmente un'architettura del caso d'uso con una visione ad alto livello, anzichè il semplice sviluppo di un software predeterminato e dal percorso strettamente imposto.

\subsection{Obiettivi personali}
L'obiettivo fondamentale dello \stage\ è colmare il divario tra il mondo accademico e quello lavorativo.
Grazie al percorso di \stage\ in una ditta esterna ho avuto l'opportunità di conoscere l'ambiente di lavoro di un'azienda nel campo \sacr{ict}, facilitandomi l'inserimento nel mondo del lavoro.

Un altro obiettivo è ottenere una formazione riguardo la tecnologia di Kafka, che ritengo possa arricchire fortemente le mie capacità e \textit{skill} professionali.
Sono pertanto interessato a sviluppare la mia formazione riguardo l'utilizzo e le implicazioni di questa tecnologia in rapida espansione, la cui formazione potrà essermi utile in molti campi anche al di fuori degli obiettivi dell'azienda ospitante lo \stage.


\section{Il percorso di Stage}


% Descrizione di come il percorso di \textit{stage} si inserisce nella visione più ampia riportata qui sopra.
%
% \bigskip\noindent
% Elenco degli obiettivi del percorso:
% \begin{itemize}
%   \item formazione riguardo Kafka e l’ambito dell’integrazione;
%   \item verificare le capacità di Kafka nell’ambito EAI;
%   \item sperimentare l’utilizzo di Kafka come \textit{Middleware} tramite una simulazione di un caso d’uso reale a servizi indipendenti.
% \end{itemize}
%
% \noindent
% Breve esposizione dei motivi che mi hanno portato a scegliere questo percorso

% TODO:
% - motivazioni riguardo il percorso di stage proposto
% - descrizione del percorso di stage tenendo conto del contesto e degli obiettivi citati sopra
% - descrivere i miei obiettivi rispetto a quelli aziendali
\subsection{Obiettivi dello \textit{stage}}

Il percorso di \textit{stage} offerto dall'azienda si inserisce all'interno della strategia aziendale più ampia descritta sopra.
Al fine di esplorare la tecnologia di Apache Kafka nell'ambito di un \textit{Middleware} per l'integrazione aziendale, l'azienda ha proposto un percorso di \textbf{\textit{stage} il cui obiettivo è la reingegnerizzazione di un flusso di dati asincrono, utilizzando un'architettura basata su Kafka all'interno di un caso d'uso simulato tramite servizi indipendenti.}

\bigskip
Lo stagista ha il compito di osservare, testare e verificare che il \software\ possa svolgere alcuni compiti inerenti all'area del \sacr{eai}, analizzando alcuni casi d'uso presenti in un \textit{Middleware} aziendale in ambito \textit{telco}.
Il percorso di prevede una durata di 300 ore lavorative.

% \bigskip\noindent
% Il percorso proposto prevede le seguenti attività e obiettivi generali:
% \begin{itemize}
%   \item formazione riguardo la piattaforma di \textit{event streaming} Kafka e l'ambito dell'integrazione aziendale;
%   \item verifica delle capacità di Kafka nell'EAI;
%   \item sperimentazione e sviluppo di un \textit{Middleware} basato su Kafka tramite una simulazione di un caso d'uso reale a servizi indipendenti.
% \end{itemize}

\subsection{Prodotti attesi}

I prodotti attesi al termine dello \textit{stage} sono dunque associati alla realizzazione di tre flussi di integrazione, basati su dei casi d’uso reali, per la gestione dei paradigmi di integrazione asincrono e asincrono con \textit{callback} (due requisiti obbligatori), e sincrono ove fosse disponibile del tempo aggiuntivo e se ritenuto opportuno; durante il percorso considerato il contesto e le opportunità offerte dal \software\ questo obiettivo verrà sostituito per testare delle funzionalità aggiuntive di Kafka.

\subsection{Contenuti formativi previsti}

La realizzazione di questi prodotti necessita una sostanziale formazione dello stagista riguardo i principali concetti del settore del \textit{Enterprise Application Integration} e l'utilizzo della piattaforma di \textit{event streaming} Kafka.
Più precisamente, i contenuti formativi previsti durante questo percorso di \textit{stage} sono i seguenti:
\begin{itemize}
  \item Concetti chiave del \gls{g_eai};
  \item \textit{Design archittetturali};
  \item Cenni di \textit{Networking} applicato alle architetture distribuite;
  \item Architetture di Integrazione e \textit{Middleware};
  \item Apache Kafka.
\end{itemize}

\subsection{Interazione tra studente e referenti aziendali}
% Personalizzare definendo le modalità di interazione col tutor aziendale
Regolarmente, (almeno una volta la settimana) ci saranno incontri online (tramite la piattaforma Google Meet) con il tutor aziendale Francesco Giovanni Sanges, il responsabile dell’area \sacr{eai} Salvatore Dore e gli esperti delle tecnologie affrontate.
I meeting saranno necessariamente online, dato il dislocamento dei vari membri in diverse città.

Lo scopo di questi incontri sarà quello di verificare lo stato di avanzamento, chiarire gli obiettivi ove necessario, affinare la ricerca e aggiornare la pianificazione iniziale.

\subsection{Pianificazione del lavoro}

Ad ogni incremento è associato un requisito obbligatorio, desiderabile o facoltativo.
A questi requisiti vi è associato un codice identificativo per favorirne il tracciamento futuro, in che precede la voce descrittiva dell'incremento.
\noindent
Ogni codice è composto da una lettera seguita da dei numeri interi, secondo il seguente modello:
\begin{center}
	\textbf{A-X.Y.Z}
\end{center}
ove, da sinistra verso destra:
\begin{itemize}

  \item \textbf{A} rappresenta la lettera che qualifica il requisito come obbligatorio, desiderabile o facoltativo, secondo la seguente notazione:
  \begin{itemize}
  	\item \textit{O} per i requisiti obbligatori, vincolanti in quanto obiettivo primario richiesto dal committente;
  	\item \textit{D} per i requisiti desiderabili, non vincolanti o strettamente necessari,
  		  ma dal riconoscibile valore aggiunto;
  	\item \textit{F} per i requisiti facoltativi, rappresentanti valore aggiunto non strettamente
  		  competitivo.
  \end{itemize}

  \item \textbf{X} rappresenta la settimana in cui viene inizialmente pianificato l'incremento (identificata da un numero incrementale e intero, partendo da 1).
  Questo consente allo studente, al tutor interno e al tutor interno una rapida quantificazione dell'avanzamento corrente dello stage rispetto a quanto inizialmente pianificato.

  \item \textbf{Y} rappresenta la posizione sequenziale prevista dell’incremento all’interno della settimana (incrementale e intero, partendo da 1). Esso è strettamente associato alla lettera.


\end{itemize}
\noindent
Di seguito viene presentata la pianificazione settimanale delle ore lavorative previste.
Ad ogni settimana sono assegnate le voci contenenti gli incrementi previsti in essa, ove i codici utilizzano la notazione descritta precedentemente.

\begin{itemize}
       \item \textbf{Prima Settimana (40 ore)}
       \begin{itemize}
           \item \textbf{O-1.1} Incontro con le persone coinvolte nel progetto per discutere i requisiti e le richieste relative al sistema da sviluppare;
           \item \textbf{O-1.2} Verifica credenziali e strumenti di lavoro assegnati;
           \item \textbf{O-1.3} Presa visione dell’infrastruttura esistente;
           \item \textbf{D-1.1} Ripasso approfondito riguardo i seguenti argomenti:
             \begin{itemize}
               \item Ingegneria del \software;
               \item Sistemi di versionamento;
               \item Architetture \software;
               \item Cenni di \textit{Networking}.
             \end{itemize}
       \end{itemize}

       \item \textbf{Seconda Settimana (40 ore)}
       \begin{itemize}
           \item \textbf{O-2.1} Nozioni fondamentali riguardo \sacr{eai} e \sacrfoot{soa};
           \item \textbf{O-2.2} Approfondimenti riguardo le Architetture a Messaggio, in particolare:
             \begin{itemize}
               \item \textit{Integration Styles};
               \item \textit{Channel Patterns};
               \item \textit{Message Construction Patterns};
               \item \textit{Routing Patterns};
               \item \textit{Transformation Patterns};
               \item \textit{System Management Patterns}.
             \end{itemize}
       \end{itemize}

       \item \textbf{Terza Settimana (40 ore)}
       \begin{itemize}
           \item \textbf{O-3.1} Apache Kafka:
             \begin{itemize}
               \item Introduzione a Kafka;
               \item Concetti fondamentali di Kafka;
               \item Avvio e \sacrfoot{cli};
               \item Programmazione in Kafka con Java.
             \end{itemize}
           \item \textbf{D-3.1} Esempi e applicazioni di Apache Kafka.
       \end{itemize}


       \item \textbf{Quarta Settimana (40 ore)}
       \begin{itemize}
           \item \textbf{O-4.1} Confluent Platform:
             \begin{itemize}
               \item Service registry;
               \item REST proxy;
               \item kSQL;
               \item Confluent connectors;
               \item Control center.
             \end{itemize}
       \end{itemize}

       \item \textbf{Quinta Settimana (40 ore)}
       \begin{itemize}
         \item \textbf{O-5.1} Analisi dei casi d'uso reali;
         \item \textbf{O-5.2} Realizzazione dei componenti per l'esecuzione dei casi di test.
       \end{itemize}

       \item \textbf{Sesta Settimana (40 ore)}
       \begin{itemize}
         \item \textbf{O-6.1} Analisi reingegnerizzazione e collaudo del flusso di integrazione asincrono.
       \end{itemize}

       \item \textbf{Settima Settimana (40 ore)}
       \begin{itemize}
           \item \textbf{O-7.1} Analisi e reingegnerizzazione e collaudo del flusso di integrazione asincrono con callback

       \end{itemize}

       \item \textbf{Ottava Settimana (20 ore)}
       \begin{itemize}
           \item \textbf{D-8.1} Analisi e reingegnerizzazione e collaudo del flusso di integrazione sincrono.
       \end{itemize}

\end{itemize}
\clearpage\noindent
Secondo questa pianificazione, le 300 ore di \stage\ previste sono approsimativamente divise in:
\begin{itemize}
  \item 160 ore di Formazione sulle tecnologie;
  \item 60 ore di Progettazione dei componenti e dei test;
  \item 60 ore di Sviluppo dei componenti e dei test;
  \item 20 ore di Valutazioni finali, Collaudo e Presentazione della Demo.
\end{itemize}

\chapter{Apache Kafka in caso d’uso simulato}

\section{Pianificazione del percorso e \textit{Way of working}}

Descrizione del Piano di Lavoro presentato, le strategie e il metodo di lavoro stabilito.

\section{Formazione e sviluppo del prodotto}

Descrizione delle fasi del percorso, con le sfide e problemi che esso ha presentato.

\section{Risultati raggiunti dal prodotto}

Confronto con i requisiti posti nel piano di lavoro, valutazione riguardo i risultati raggiunti, quelli non raggiunti e i possibili sviluppi futuri

\chapter{Valutazione retrospettiva}

% \section{Obiettivi soddisfatti dallo stage}
%
% Valutazione oggettiva riguardo il percorso e i risultati raggiunti da esso.
%
% \section{Maturazione professionale acquisita}
%
% Descrizione delle conoscenze e abilità professionali acquisite grazie al percorso di \textit{stage}.
% Valutazione del miglioramento personale portato avanti durante il percorso di stage.
%
%
% \section{Distanza tra le competenze necessarie e quelle acquisite nel corso di studi}
%
% Breve valutazione riguardo le difficoltà riscontrate, e considerazioni riguardo le competenze ottenute durante il corso di laurea che più mi hanno aiutato durante il percorso.

\section{Obiettivi dello stage raggiunti}

Gli obiettivi principali dello stage sono stati raggiunti con successo.

I microservizi che compongono il prodotto finale hanno raggiunto efficacemente il risultato preposto, creando il sistema richiesto dalla sperimentazione; i due servizi di test quali \sacr{ws} \textit{Client} e \sacr{ws} \textit{Provider} si scambiano messaggi tramite un \middleware\ basato su Apache Kafka.

La sperimentazione ha testato alcune delle capacità di Apache Kafka con esito positivo, fornendo le basi per ulteriori percorsi di approfondimento che possono portare all'implementazione della piattaforma di \textit{event streaming} all'interno degli attuali sistemi di integrazione con il ruolo di \middleware.

Un possibile percorso potrebbe ad esempio modellare e sviluppare un caso d'uso molto più complesso, con simulazione di un flusso di dati continuo e di grandi dimensioni, con dati provenienti da fonti multiple, un numero maggiore di \textit{producer} e \textit{consumer}, e la sperimentazione di ulteriori funzionalità presenti nei \middleware\ attualmente utilizzati.

\noindent
Di seguito viene ripresa parte della tabella \ref{tab:pianificazione} vista nella sezione \ref{sec:pianificazione}.

\onehalfspacing
\begin{small}
  \begin{center}
    \centering
    \renewcommand\arraystretch{1.6}
    \begin{longtable}{| >{\centering\arraybackslash}m{2cm}|m{9.5cm}|>{\centering\arraybackslash}m{2.2cm}|}
      \hline
      \textsc{\textbf{Requisito}} & \textsc{\textbf{Task associati}} & \textsc{\textbf{Completato}} \\
      \hline
      O-5.1 & Analisi dei casi d'uso reali & \textsc{si} \\
      \hline
      O-5.2 & Realizzazione dei componenti per l'esecuzione dei casi di test & \textsc{si}\\
      \Xhline{2\arrayrulewidth}
      O-6.1 & Analisi re-ingegnerizzazione e collaudo del flusso di integrazione asincrono & \textsc{si} \\
      \Xhline{2\arrayrulewidth}
      O-7.1 & Analisi e re-ingegnerizzazione e collaudo del flusso di integrazione asincrono con callback & \textsc{si}\\
      \Xhline{2\arrayrulewidth}
      O-8.1 & Analisi e re-ingegnerizzazione e collaudo del flusso di integrazione sincrono & \textsc{no}\\
      \hline
      \textsc{Non previsto} & Sperimentazione di funzioni aggiuntive: protezione di un dato sensibile & \textsc{si}\\
      \hline

      \caption{Obiettivi dello stage raggiunti}
    \end{longtable}
  \end{center}
\end{small}
% \doublespacing

Come detto in precedenza, il requisito O-8.1 è stato scartato in favore della sperimentazione di alcune funzioni aggiuntive di Kafka, inizialmente non pianificate.

% \section{Prodotti sviluppati}
%
% Il prodotto finale del progetto è composto da i diversi componenti illustrati nella sottosezione \ref{sub:uml_component}, per un totale di dieci servizi ognuno nel proprio
% \textit{container} Docker (conformi con il \textit{deployment diagram} alla sottosezione \ref{sub:uml_deployment}).
%
% Questi microservizi hanno raggiunto efficacemente il risultato preposto, creando il sistema richiesto dalla sperimentazione; i due servizi di test quali \sacr{ws} \textit{Client} e \sacr{ws} \textit{Provider} si scambiano messaggi tramite un \middleware\ basato su Apache Kafka.

\section{Contenuti formativi acquisiti}

Il processo di formazione ha riguardato principalmente i concetti riguardo il settore del \textit{\acrlong{eai}} e le tecnologie legate ad Apache Kafka.

Di seguito espongo i requisiti formativi soddisfatti, in relazione al piano di lavoro iniziale.

\onehalfspacing
\begin{small}
  \begin{center}
    \centering
    \renewcommand\arraystretch{1.6}
    \begin{longtable}{| >{\centering\arraybackslash}m{2cm}|m{9.5cm}|>{\centering\arraybackslash}m{2.2cm}|}
      \hline
      \textsc{\textbf{Requisito}} & \textsc{\textbf{Task associati}} & \textsc{\textbf{Completato}} \\
      \hline
     %  O-1.1 & Incontro con le persone coinvolte nel progetto per discutere i requisiti e le richieste relative al sistema da sviluppare & \textsc{si} \\
     %  \hline
     %  O-1.2 & Verifica credenziali e strumenti di lavoro assegnati & \textsc{si}\\
     %  \hline
     %  O-1.3 & Presa visione dell’infrastruttura esistente & \textsc{si}\\
     %  \hline
     %  D-1.1 & Ripasso approfondito riguardo i seguenti argomenti:
     %    \smallskip
     %    \begin{itemize}
     %       \item Ingegneria del \software;
     %       \item Sistemi di versionamento;
     %       \item Architetture \software;
     %       \item Cenni di \textit{Networking}.
     %     \end{itemize} &  \textsc{si}\\
     % \Xhline{2\arrayrulewidth}
     O-2.1 & Nozioni fondamentali riguardo \sacr{eai} e \sacr{soa} & \textsc{si}\\
     \hline
     O-2.2 & Approfondimenti riguardo le Architetture a Messaggio, in particolare:
       \begin{itemize}
          \item \textit{Integration Styles};
          \item \textit{Channel Patterns};
          \item \textit{Message Construction Patterns};
          \item \textit{Routing Patterns};
          \item \textit{Transformation Patterns};
          \item \textit{System Management Patterns}.
        \end{itemize} & \textsc{si}\\
    \Xhline{2\arrayrulewidth}

    O-3.1 & Apache Kafka:
      \begin{itemize}
          \item Introduzione a Kafka;
          \item Concetti fondamentali di Kafka;
          \item Avvio e \sacr{cli};
          \item Programmazione in Kafka con Java.
        \end{itemize} & \textsc{si}\\
    \hline
    D-3.1 & Esempi e applicazioni di Apache Kafka & \textsc{si} \\
    \Xhline{2\arrayrulewidth}

    O-4.1 & Confluent Platform:
      \begin{itemize}
          \item \textit{Service registry};
          \item \sacr{rest} \textit{proxy};
          \item kSQL;
          \item Confluent \textit{connectors};
          \item \textit{Control center}.
      \end{itemize} & \textsc{si}\\
    \Xhline{2\arrayrulewidth}


      \caption{Contenuti formativi acquisiti}
    \end{longtable}
  \end{center}
\end{small}
% \doublespacing

\section{Obiettivi personali raggiunti}
%
% - molto soddisfacenti
%
% - espansione delle conoscenze tecnologiche
%
% - maturazione professionale
%
% - inserimento in un contesto Aziendale
%
% - gestione e organizzazione di progetto
Valuto i risultati personali raggiunti dal percorso di \stage\ molto soddisfacenti dal punto di vista di una maturazione professionale.

La formazione ricevuta affine al settore del \textit{\acrlong{eai}} ha allargato le mie conoscenze tecnologiche nell'ambito dell'ingegneria del \software\, in particolare ho apprezzato l'approfondimento riguardo le architecture \software\ moderne e i sistemi di integrazione associati al mondo del \textit{Big Data}.
Ho apprezzato molto lo studio delle tecnologie emergenti per un'innovazione aziendale nel \sacr{eai}, quali Apache Kafka e Confluent, e le conseguenze importanti dovute alla migrazione di un sistema verso una \acrlong{eda}.

L'alto livello di organizzazione personale che ho tenuto durante il percorso ha garantito un buona qualità nel \textit{way of working}, preparandomi all'inserimento nel mondo del lavoro e a un contesto aziendale innovativo e all'avanguardia.

Gli esperti aziendali che mi hanno fornito il supporto e le linee guida necessarie al compimento dello \stage\ sono stati per me una grande fonte di apprendimento; in particolare, ho imparato gli importanti passi e considerazioni necessarie che guidano la sperimentazione associata all'implementazione di tecnologie innovative, sia in ambito tecnico che architetturale.

\begin{figure}[h]
  \begin{center}
    \includegraphics[width=0.85\textwidth]{images/pdca.png}
    \caption{\acrlong{pdca}}
    \captionsetup{aboveskip=2pt}
    \caption*{\begin{footnotesize}\textit{Fonte:} \url{http://quickstart-indonesia.com/siklus-pdca/}\end{footnotesize}}
  \end{center}
\end{figure}

L'adozione del \sacrfoot{pdca} (figura \thefigure) nel mio metodo di lavoro è stato essenziale per dei miglioramenti personali in diversi ambiti, soprattutto quelli legati alla gestione del tempo (come visto nella sotto-sezione \ref{subsub:auto-miglioramento}), efficienza organizzativa ed efficacia di sviluppo.

\section{Distanza rispetto ai contenuti del corso di studi}

Nonostante le tecnologie e i concetti appresi siano totalmente nuovi rispetto agli insegnamenti del corso, quest'ultimi hanno fornito la preparazione necessaria all'apprendimento rapido di tali novità; in un settore lavorativo in rapida evoluzione come quello dell'informatica, la capacità di adattamento all'utilizzo di strumenti moderni come Apache Kafka e quella di comprendere facilmente le architetture innovative è di fondamentale importanza.

I risultati dei vari corsi di studi hanno esplicitato la loro utilità nello \stage, culmine del percorso che porta alla Laurea in Informatica.

Alcuni corsi in particolare hanno provveduto sostanzialmente la distanza tra l'ambiente universitario e quello lavorativo, quali ingegneria del software (con progetto associato), basi di dati, tecnologie web, e i vari corsi di programmazione.

Riassumendo, il corso di studi ha pienamente soddisfatto i requisiti di cui il mio percorso di \stage\ ha necessitato e dato una rapida accelerazione alla mia carriera lavorativa.


% \printbibliography
% \clearpage{\pagestyle{plain}\cleardoublepage} %Numerazione araba per i capitoli
% \pagenumbering{arabic}
%
%
% \clearpage{\pagestyle{plain}\cleardoublepage} %Comando per iniziare il capitolo su pagina dispari
% \chapter{Primo Capitolo} %Nome capitolo
% \label{chapter:primo_capitolo} %Label per creare riferimenti al capitolo
% \input{intro_cap1.tex} %File in cui verrà scritto il capitolo
%
% \clearpage{\pagestyle{plain}\cleardoublepage}
% \chapter{Immagini e Tabelle}
% \label{chapter:immagini_e_tabelle}
% \input{intro_cap2.tex}
%
% \clearpage{\pagestyle{plain}\cleardoublepage}
% \chapter{Formule}
% \label{chapter:formule}
% \input{intro_cap3.tex}
%
% \clearpage{\pagestyle{plain}\cleardoublepage}
% \chapter{Pseudocodice e codice}
% \label{chapter:codice}
% \input{intro_cap4.tex}
%
% \clearpage{\pagestyle{plain}\cleardoublepage}
% \input{bibliografia.tex}
\end{document}
